% -*- coding: utf-8 -*-
\documentclass{ctexart}
\usepackage[margin = 1 in]{geometry}
\usepackage[utf8]{inputenc}
\usepackage[T1]{fontenc}
\usepackage{amsmath}
\usepackage{enumitem}
\usepackage{amsfonts}
\usepackage{amssymb}
\usepackage{amsthm}
\usepackage{stmaryrd}
\usepackage{hyperref}
\hypersetup{colorlinks=true, linkcolor=blue, filecolor=magenta, urlcolor=cyan,}
\urlstyle{same}
\usepackage{graphicx}
\usepackage[export]{adjustbox}
\usepackage{mdframed}
\usepackage{booktabs,array,multirow}
\usepackage{esint}
\usepackage{xeCJK}
\usepackage{adjustbox}
\newcommand{\HRule}{\begin{center}\rule{0.5\linewidth}{0.2mm}\end{center}}
\graphicspath{ {./images/} }
\newcommand{\customfootnote}[1]{
  \let\thefootnote\relax\footnotetext{#1}
}

\usepackage{titlesec} % 用于控制标题格式
\usepackage{etoolbox} % 用于钩子机制

\newcommand{\cplus}{\raisebox{-1ex}{+}}


\newtheorem{theorem}{\indent 定理}[section]
\newtheorem{lemma}[theorem]{\indent 引理}
\newtheorem{proposition}[theorem]{\indent 命题}
\newtheorem{corollary}[theorem]{\indent 系}
\newtheorem{definition}{\indent 定义}[section]
\newtheorem{example}{\indent 例}[section]
\newtheorem{remark}{\indent 注}[section]
\newenvironment{solution}{\begin{proof}[\indent\bf 解]}{\end{proof}}
\renewcommand{\proofname}{\indent\bf 证明}


\makeatletter
% 重新定义 TOC 前的标题部分
\renewcommand*\contentsname{目次}
% 在 TOC 前添加居中设置
%\preto\tableofcontents{\clearpage\section*{\centering\contentsname}}
\makeatother


\title{从祖冲之的圆周率谈起}
\author{华罗庚}
\date{1962年}
\begin{document}
\maketitle

\newpage
\section*{编者的话}

数学课外读物对于帮助学生学好数学, 扩大他们的数学知识领域, 是很有好处的.近年来, 越来越多的中学学生和教师, 都追切希望出版更多的适合中学生阅读的通俗数学读物. 我们约请一些数学工作者, 编了这套“数学小丛书”, 陆续分册出版, 来适应这个要求.

这套书打算介绍一些课外的数学知识, 以扩大学生的知识领域, 加深对数学基础知识的掌握, 引导学生独立思考, 理论联系实际.

这是我们的初步想法和尝试. 热切地希望数学工作者和读者对我们的工作提出宝贵的意见和建议, 更希望数学工作者为中学生写出更多更好的数学课外读物.

\rightline{北京市数学会}

\rightline{1962 年 4 月}
\newpage
\begin{quotation}
  \textit{“……宋末南徐州从事史祖冲之更开密法, 以圆径……为一丈, 圆周盈数三丈一尺四寸一分五厘九毫二秒七忽; 胰数三丈一尺四寸一分五厘九毫二秒六忽;正数在盈朒二限之间. 密率。圆径一百一十三,圆周三百五十五;约率、圆径七,周二十二.……指要精密, 算氏之最者也. 所著书, 名为缀术, 学官其能究其深臭, 是故废而不理."}

\rightline{——唐长孙无忌《隋书》卷十六律历卷十一——}
\end{quotation}

\clearpage

\tableofcontents

\newpage
\section{祖冲之的约率 \(\frac{22}{7}\) 和密率 \(\frac{355}{113}\)}

祖冲之是我国古代的伟大数学家. 他生于公元 429 年, 卒于公元 500 年. 他的儿子祖喧和他的孙子祖皓, 也都是数学家, 善算历.

关于圆周率 \(\pi\) ,祖冲之的贡献有二:

\begin{enumerate}[label=\textup{(\roman*)}]
\item\label{enu:1} \({3.1415928} < x < {3.1415927}\) ,
\item\label{enu:2} 他用 \(\frac{22}{7}\) 作为约率, \(\frac{355}{113}\) 作为密率.
\end{enumerate}

这些结果是刘徽割圆术之后的重要发展. 刘徽从圆内接正六边形起算, 令边数一倍一倍地增加, 即 12、24、48、96、…… 1586 ,……,因而逐个算出六边形、十二边形、二十四边形…… 的面积, 这些数值逐步地逼近圆周率. 刘徽方法的特点, 是得出一批一个大于一个的数值, 这样来一步一步地逼近圆周率. 这方法是可以无限精密地逼近圆周率的, 但每一次都比圆周率小.

祖冲之的结果\ref{enu:1}从上下两方面指出了圆周率的误差范围. 这是大家都容易看到的事实, 因此在这本小书中不预备多讲. 我只准备着重地谈一谈结果 \ref{enu:2}. 在谈到 \(\frac{355}{113}\) 的时候, 一定能从

\[
\frac{355}{113} = {3.1415929}\cdots
\]

看出,他所提出的 \(\frac{355}{113}\) 惊人精密地接近于圆周率,准确到六位小数. 也有人会指出这一发现比欧洲人早了一千年. 因为德国人奥托 (Valenlinus Otto) 在 1573 年才发现这个分数. 如果更深入地想一下,就会发现 \(\frac{22}{7}\) 和 \(\frac{355}{113}\) 的意义还远不止这些. 有些人认为那时的人们喜欢用分数来计算. 这样看问题未免太简单了. 其实其中孕育着不少道理, 这道理可以用来推算天文上的很多现象. 无怪乎祖冲之祖孙三代都是算历的专家. 这个约率和密率, 提出了 “用有理数最佳逼近实数” 的问题. “逼近”这个概念在近代数学中是十分重要的.

\section{人造行星将于 2113 年又接近地球}

我们暂且把“用有理数最佳逼近实数”的问题放一放, 而再提一个事实:

1959年苏联第一次发射了一个人造行星, 报上说: 苏联某专家算出, 五年后这个人造行星 又将接近地球, 在 2113 年又将非常接近地球. 这是怎样算出来的? 难不难, 深奥不深奥? 我们中学生能懂不能懂? 我说能懂的: 不需要专家, 中学生是可以学懂这个方法的.

先看为什么五年后这个人造行星会接近地球. 报上登过这个人造行星绕太阳一周的时间是 450 天. 如果以地球绕日一周 360 天计算, 地球走五圈和人造行星走四圈不都是 1800 天吗? 因此五年后地球和人造行星将相互接近. 至于为什么在 2113 年这个人造行星和地球又将非常接近? 我们将在第四节中说明.

再看五圈是怎样算出来的. 任何中学生都会回答: 这是由于约分

\[
\frac{360}{450} = \frac{4}{5}
\]

而得来的, 或者这是求 450 和 360 的最小公倍数而得来的. 它们的最小公倍数是 1800,而 \(\frac{1800}{360} = 5,\frac{1800}{450} = 4\) ; 也就是当地球绕太阳五圈时, 人造行星恰好回到了原来的位置. 求最小公倍数在这儿找到了用场. 在进入下节介绍辗转相除法之前,我们再说一句,地球绕太阳并不是 360 天一周,而是 \({365}\frac{1}{4}\) 天. 因而仅仅学会求最小公倍数法还不能够应付这一问题, 还须更上一层楼.

\section{辗转相除法和连分数}

我们还是循序渐进吧. 先从简单的(原来在小学或初中一年级讲授的辗转相除法讲起. 但我们采用较高的形式, 采用学过代数学的同学所能理解的形式.

给两个正整数 \(a\) 和 \(b\) ,用 \(b\) 除 \(a\) 得商 \({a}_{0}\) ,余数 \(r\) ,写成式子
\begin{equation}
\label{eq:1}
a = {a}_{0}b + r,\;0 \leq r < b. 
\end{equation}


这是最基本的式子. 如果 \(r\) 等于 0,那么 \(b\) 可以除尽 \(a\) ,而 \(a\) 、 \(b\) 的最大公约数就是 \(b\) .

如果 \(r \neq 0\) ,再用 \(r\) 除 \(b\) ,得商 \({a}_{1}\) ,余数 \({r}_{1}\) ,即
\begin{equation}
\label{eq:2}
b = {a}_{1}r + {r}_{1}\;0 \leq {r}_{1} < r. 
\end{equation}

如果 \({r}_{1} = 0\) ,那么 \(r\) 除尽 \(b\) ,由 ( 1 ) 它也除尽 \(a\) . 又任何一个除尽 \(a\) 和 \(b\) 的数,由 ( 1 ) 也一定除尽 \(r\) . 因此, \(r\) 是 \(a\text{、}b\) 的最大公约数.

如果 \({r}_{1} \neq 0\) ,用 \({r}_{1}\) 除 \(r\) ,得商 \({a}_{2}\) ,余数 \({r}_{2}\) ,即
\begin{equation}
\label{eq:3}
r = {a}_{2}{r}_{1} + {r}_{2},\;0 \leq {r}_{2} < {r}_{1}.
\end{equation}

如果 \({r}_{2} = 0\) ,那么由~\eqref{eq:2} \({r}_{1}\) 是 \(b,r\) 的公约数,由~\eqref{eq:1}它也是 \(a\text{、}b\) 的公约数. 反之,如果一数除得尽 \(a\text{、}b\) ,那由~\eqref{eq:1}它一定除得尽 \(b\text{、}r\) ,由~\eqref{eq:2}它一定除得尽 \(r\text{、}{r}_{1}\) ,所以 \({r}_{1}\) 是 \(a\text{、}b\) 的最大公约数.

如果 \({r}_{2} \neq 0\) ,再用 \({r}_{2}\) 除 \({r}_{1}\) ,如法进行. 由于 \(b > r > {r}_{1}\) \(> {r}_{2} > \cdots \cdots \left( { \geq 0}\right)\) 逐步小下来,因此经过有限步骤后一定可以找出 \(a\text{、}b\) 的最大公约数来 (最大公约数可以是 1). 这就是辗转相除法, 或称欧几里得算法. 这个方法是我们这本小册子的灵魂.

\begin{example}
  求 360 和 450 的最大公约数.
\[
{450} = 1 \times {360} + {90}
\]

\[
{360} = 4 \times {90}\text{.}
\]

所以 90 是 360、450 的最大公约数. 由于最小公倍数等于两数相乘再除以最大公约数, 因此这二数的最小公倍数等于

\[
{360} \times {450} \div {90} = {1800},
\]

因而得出上节的结果.
\end{example}

\begin{example}
  求 42897 和 18644 的最大公约数.

\[
{42897} = 2 \times {18644} + {5609},
\]

\[
{18644} = 3 \times {5609} + {1817},
\]

\[
{5609} = 3 \times {1817} + {158},
\]

\[
{1817} = {11} \times {158} + {79},
\]

\[
{158} = 2 \times {79}\text{. }
\]

因此最大公约数等于 79 .

\end{example}

计算的草式如下:
计算的草式如下:

\begin{table}[h]
 \centering
\begin{tabular}{r|l|l}
42897 &    &       \\
$-$ 37288 & 2  & 18644 \\ \cline{1-1}
5609  & 3  & 16827 \\ \cline{3-3} 
5451  & 3  & 1817  \\ \cline{1-1}
158   & 11 & 1738  \\ \cline{3-3} 
158   & 2  & 79    \\ \cline{1-1}
0     &    &      
\end{tabular}
\end{table}

例 2 的计算也可以写成为
\begin{align*}
\cfrac{42897}{18644} = &2 + \cfrac{5699}{18644} = 2 + \cfrac{1}{\cfrac{18644}{5609}}\\
= &2 + \cfrac{1}{3 + \cfrac{1817}{5609}} = 2 + \cfrac{1}{3 + \cfrac{1}{3 + \cfrac{188}{1817}}}\\
= &2 + \cfrac{1}{3 + \cfrac{1}{3 + \cfrac{1}{{11} + \cfrac{79}{158}}}}\\
= &2 + \cfrac{1}{3 + \cfrac{1}{3 + \cfrac{1}{{11} + \cfrac{1}{2}}}}
\end{align*}


这样的繁分数称为连分数. 为了节省篇幅, 我们把它写成

\[
2 + \frac{1}{3} \cplus \frac{1}{3} \cplus \frac{1}{11} \cplus \frac{1}{2}
\]

注意 \(2\text{、}3\text{、}3\text{、}{11}\text{、}2\) 都是草式中间一行的数字. 倒算回去,得
\begin{align*}
  &2 + \frac{1}{3} \cplus \frac{1}{3} \cplus \frac{1}{11} \cplus \frac{1}{2}\\
    = &2 + \frac{1}{3} \cplus \frac{1}{3} \cplus \frac{2}{23}\\
= &2 + \frac{1}{3} \cplus \frac{23}{71} = 2 + \frac{71}{236} = \frac{543}{298}.
\end{align*}

这就是原来分数的既约分数.

依次截段得

\[
2,\;2 + \frac{1}{3} = \frac{7}{3},\;2 + \frac{1}{3} \cplus \frac{1}{3} = \frac{23}{10},\;2 + \frac{1}{3} \cplus \frac{1}{3} \cplus \frac{1}{11} = \frac{260}{113}.
\]

这些分数称为 \(\frac{543}{236}\) 的渐近分数. 我们看到第一个渐近分数比 \(\frac{543}{236}\) 小, 第二个渐近分数比它大, 第三个又比它小, ……为什么叫做渐近分数? 我们看一下分母不超过 10 的分数和 \(\frac{543}{236}\) 相接近的情况.

分母是 \(1,2,3,4,5,6,7,8,9,{10}\) ,而最接近于 \(\frac{543}{236}\) 的分数

是

\[
\frac{2}{1},\frac{5}{2},\frac{7}{3},\frac{9}{4},\frac{12}{5},\frac{14}{6},\frac{16}{7},\frac{19}{8},\frac{21}{9},\frac{23}{10}
\]

取二位小数, 它们分别等于

\[{2.00},{2.50},{2.33},{2.25},{2.40},{2.33},{2.29},{2.38},{2.33},{2.30}\]

和 \(\frac{543}{236} = {2.30}\) 相比较,可以发现其中有几个特出的既约分数

\[
\frac{2}{1},\frac{5}{2},\frac{7}{3},\frac{16}{7},\frac{23}{10}
\]

这几个数比它们以前的数都更接近于 \(\frac{643}{236}\) . 而其中 \(\frac{2}{1},\frac{7}{3},\frac{23}{10}\) 都是由连分数截段算出的数, 即它们都是渐近分数.

我们现在再证明, 分母小于 113 的分数里面, 没有一个比 \(\frac{260}{113}\) 更接近于 \(\frac{543}{236}\) 了. 要证明这点很容易,首先

\[
\left| {\frac{543}{236} - \frac{260}{113}}\right| = \frac{1}{{236} \times {113}}
\]

命 \(\frac{a}{b}\) 是任一分母 \(b\) 小于 113 的分数,那么

\[
\left| {\frac{543}{23a} - \frac{a}{b}}\right| = \frac{\left| {543}b - {236}a\right| }{{236} \times b} \geq \frac{1}{{236} \times b} > \frac{1}{{236} \times {113}}.
\]

\section{答第二节的问}

现在我们来回答第二节里的问题:怎样算出人造行星 2113 年又将非常接近地球?

人造行星绕日一周需 450 天,地球绕日一周是 \({355}\frac{1}{4}\) 天. 如果以 \(\frac{1}{4}\) 天做单位,那么人造行星和地球绕日一周的时间各为 1800 和 1461 个单位。如上节所讲的方法,
\begin{table}[h]
  \centering
\begin{tabular}{r|l|l}
1800 &   &      \\
1461 & 1 & 1461 \\ \cline{1-1}
339  & 4 & 1356 \\ \cline{3-3} 
315  & 3 & 105  \\ \cline{1-1}
24   & 4 & 96   \\ \cline{3-3} 
18   & 2 & 9    \\ \cline{1-1}
6    & 1 & 6    \\ \cline{3-3} 
6    & 2 & 3    \\ \cline{1-1}
0    &   &     
\end{tabular}
\end{table}

即得连分数

\[
1 + \frac{1}{4} \cplus \frac{1}{3} \cplus \frac{1}{4} \cplus \frac{1}{2} \cplus \frac{1}{1} \cplus \frac{1}{2}
\]

由此得渐近分数

\[
1,1 + \frac{1}{4} = \frac{5}{4},1 + \frac{1}{4} \cplus \frac{1}{3} = \frac{16}{13},1 + \frac{1}{4} \cplus \frac{1}{3} \cplus \frac{1}{4} = \frac{69}{56},
\]

\[
1 + \frac{1}{4} \cplus \frac{1}{3} \cplus \frac{1}{4} \cplus \frac{1}{2} = \frac{154}{125},\cdots \cdots
\]

第一个渐近分数说明了地球 5 圈, 人造行星 4 圈, 即五年后人造行星和地球接近. 但地球 16 圈, 人造行星 13 圈更接近些; 地球 69 圈, 人造行星 56 圈还要接近些; 而地球 154 圈, 人造行星 125 圈又更要接近些. 这就是报上所登的苏联专家所算出的数字了, 这也就是在

\[
{1059} + {154} = {2113}
\]

年, 人造行星将非常接近地球的道理.

当然, 由于连分数还可以做下去, 所以我们可以更精密地算下去,但是因为 450 天 和 365 \(\frac{1}{4}\) 天这两个数字并不很精确,所以再继续算下去也就没有太大的必要了. 但读者不妨作为习题再算上一项.

\section{约率和密率的内在意义}

在上节中,我们将 \({365}\frac{1}{4}\text{、}{450}\) 乘 4 以后再算. 实际上, 在求两个分数的比的连分数时, 不必把它们化为两个整数再算.

例如, 3.14159265 和 1 可以计算如下:
\begin{table}[h]
  \centering
\begin{tabular}{r|l|l}
3.1415926 &   &      \\
3 & 3 & 1 \\ \cline{1-1}
0.1415926  & 7 & 0.99114855 \\ \cline{3-3} 
0.13277175  & 15 & 0.00885145  \\ \cline{1-1}
0.00882090   & 1 & 0.00882090   \\ \cline{3-3} 
   &  & 0.00003055    \\ 
\end{tabular}
\end{table}

\[
\text{即得:\qquad}x = 3 + \frac{1}{7} \cplus \frac{1}{15} \cplus \frac{1}{1} + \cdots
\]

渐近分数是
\begin{center}
  \begin{table}[h]
\begin{tabular}{ll}
3                                                                  & 【径一周三, 《周髀算径》】          \\
$3 + \frac{1}{7} = \frac{22}{7}$                                   & 【约率, 何承天 公元 370-447】    \\
$3 + \frac{1}{7} \cplus \frac{1}{15} = \frac{333}{106}$                 &                         \\
\(3 + \frac{1}{7} \cplus \frac{1}{10} \cplus \frac{1}{1} = \frac{355}{113}\) & 【密率, 祖冲之 (公元 429-500 )】
\end{tabular}
\end{table}
\end{center}


实际算出 \(\frac{22}{7} = {3.142}\) 和 \(\frac{355}{113} = {3.1415929}\) , 误差分别在小数点后第三位和第七位.

用比 \(\pi = {3.14159265}\) 更精密的圆周率来补算,我们可以得出

\[
x = 3 + \frac{1}{7} \cplus \frac{1}{15} \cplus \frac{1}{1} \cplus \frac{1}{292} \cplus \frac{1}{1} \cplus \frac{1}{1} + \cdots
\]

\(\frac{355}{113}\) 之后的一个渐近分数是 \(\frac{103993}{33102}\) . 这是一个很不容易记忆、也不便于应用的数.

以下的数据说明,分母比 7 小的分数不比 \(\frac{22}{7}\) 更接近于 \(x\) , 而分母等于 8 的也不比 \(\frac{22}{7}\) 更接近于 \(\pi\) .

\begin{center}
\adjustbox{max width=\textwidth}{
\begin{tabular}{|c|c|c|c|}
\hline
分母 \(q\) & $qx$  & 分子 \(p\) & \(x = \frac{p}{q}\) \\
\hline
1 & 3.1416 & 3 & 0.1416 \\
\hline
2 & 6.2832 & 6 & 0.1416 \\
\hline
3 & 9.4248 & 9 & 0.1416 \\
\hline
4 & 12.5664 & 13 & -0.1084 \\
\hline
5 & 15.7080 & 16 & -0.0584 \\
\hline
6 & 18.8496 & 19 & -0.0251 \\
\hline
7 & 21.9912 & 22 & -0.0013 \\
\hline
8 & 25.1328 & 25 & 0.0166 \\
\hline
\end{tabular}
}
\end{center}

关于 \(\frac{333}{106}\) 也有同样性质 (以后将会证明的). 为了避免不必要的计算, 我仅仅指出,

\[
\left| {\pi - \frac{330}{105}}\right| = \left| {\pi - \frac{22}{7}}\right| = {0.0013},
\]

\[
\left| {\pi - \frac{333}{106}}\right| = {0.00009}
\]

\[
\left| {\pi - \frac{336}{107}}\right| = {0.0014}
\]

以 \(\frac{333}{106}\) 的误差为最小. 又

\[
\left| {\pi - \frac{352}{112}}\right| = {0.0013}
\]

\[
\left| {\pi - \frac{355}{113}}\right| = {0.0000007}
\]

\[
\left| {\pi - \frac{358}{114}}\right| = {0.0012}
\]

以 \(\frac{355}{113}\) 的误差为最小.

总之,在分母不比 \(8\text{、}{107}\text{、}{114}\) 大的分数中,分别不比 \(\frac{22}{7}\) . \(\frac{333}{106}\text{、}\frac{335}{113}\) 更接近于 \(\pi\) ; 而 \(\frac{22}{7}\text{、}\frac{355}{113}\) 又是两个相当便于记忆和应用的分数. 我国古代的数学家祖冲之能在这么早的年代, 得到 \(\pi\) 的这样两个很理想的近似值,是多么不简单的事.

\paragraph*{注意} 并不是仅有这些数有这性质,例如 \(\frac{331}{99}\) 就是一个.

\[
\left| {\pi - \frac{308}{98}}\right| = {0.0013},\;\left| {\pi - \frac{311}{99}}\right| = {0.0002},
\]

\[
\left| {\pi - \frac{314}{100}}\right| = {0.0016}\text{.}
\]

又

\[
\frac{374}{119} = {3.1429},\;\frac{377}{120} = {3.14167},\;\frac{380}{121} = {3.1405}.
\]

这说明 \(\frac{377}{120}\) 比另外两个数来得好,但是它的分母比 \(\frac{355}{113}\) 的分母大,而且它不比 \(\frac{355}{113}\) 更精密,它的精密度甚至落后于 \(\frac{333}{106}\) .

\section{为什么四年一闰, 而百年又少一闰?}

如果地球绕太阳一周是 365 天整, 那我们就不需要分平年和闰年了. 也就是没有必要每隔四年把二月份的 28 天改为 29 天了.

如果地球绕太阳一周恰恰是 \({365}\frac{1}{4}\) 天,那我们四年加一天的算法就很精确, 没有必要每隔一百年又少加一天了.

如果地球绕太阳一周恰恰是 365.24 天, 那一百年必须有 24 个闰年, 即四年一闰而百年少一闻, 这就是我们用的历法的来源. 由 \(\frac{1}{4}\) 可知: 每四 (分母) 年加一 (分子) 天; 由 \(\frac{24}{100}\) 可知: 每百(分母) 年加 24 (分子) 天.

但是事实并不这样简单, 地球绕日一周的时间是 365.2422 天. 由

\[
{0.2422} = \frac{2422}{10000}
\]

可知: 一万年应加上 2422 天, 但按百年 24 闰计算只加了 2400 天, 显然少算了 22 天.

现在让我们用求连分数的渐近分数来求得更精密的结果.

我们知道地球绕太阳一周需时 365 天 5 小时 48 分 46 秒. 也当随

\[
{365} + \frac{5}{24} + \frac{48}{{24} \times {60}} + \frac{46}{{24} \times {60} \times {60}} = {365}\frac{10463}{43200}
\]

展开得连分数

\[
{365}\frac{10463}{43200} = {365} + \frac{1}{4} \cplus \frac{1}{7} \cplus \frac{1}{1} \cplus \frac{1}{3} \cplus \frac{1}{5} \cplus \frac{1}{64}.
\]

算法是

\begin{table}[h]
  \centering
\begin{tabular}{r|l|l}
43200 &   &      \\
41852 & 4 & 10463 \\ \cline{1-1}
1348  & 7 & 9436 \\ \cline{3-3} 
1027  & 1 & 1027  \\ \cline{1-1}
321   & 3 & 963   \\ \cline{3-3} 
320   & 5 & 64    \\ \cline{1-1}
1    & 64 & 64    \\ \cline{3-3} 
      &  & 0    
\end{tabular}
\end{table}
分数部分的渐近分数是

\[
\frac{1}{4},\frac{1}{4} \cplus \frac{1}{7} = \frac{7}{29},\frac{1}{4} \cplus \frac{1}{7} \cplus \frac{1}{1} = \frac{8}{33},\frac{1}{4} \cplus \frac{1}{7} \cplus \frac{1}{1} \cplus \frac{1}{3} = \frac{31}{128}\text{,}
\]

\[
\frac{1}{4} \cplus \frac{1}{7} \cplus \frac{1}{1} \cplus \frac{1}{3} \cplus \frac{1}{5} = \frac{163}{673}
\]

\[
\frac{1}{4} \cplus \frac{1}{7} \cplus \frac{1}{1} \cplus \frac{1}{3} \cplus \frac{1}{5} \cplus \frac{1}{64} = \frac{10463}{43200}
\]

和 \(\pi\) 的渐近分数一样,这些渐近分数也一个比一个精密. 这说明四年加一天是初步的最好的近似值, 但 29 年加 7 天更精密些, 33 年加 8 天又更精密些, 而 99 年加 24 天正是我们百年少一闰的由来. 由数据也可见 128 年加 31 天更精密(也就是说头三个 33 年各加 8 天,后一个 29 年加 7 天,共 \(3 \times {33} + {29} = 128 \)  年加 \(3 \times 8 + 7 = {31}\) 天),等等.

所以积少成多, 如果过了 43200 年, 照百年 24 闰的算法一共加了 \({432} \times {24} = {10368}\) 天,但是照精密的计算,却应当加 10463 天, 一共少加了 95 天. 也就是说, 按照百年 24 闰的算法, 过 43200 年后, 人们将提前 95 天过年, 也就是在秋初就要过年了!

不过我们的历法除订定四年一闰、百年少一闰外, 还订定每 400 年又加一闰, 这就差不多补偿了按百年 24 闰计算少算的差数. 因此照我们的历法, 即使过 43200 年后, 人们也不会在秋初就过年. 我们的历法是相当精确的.

\section{农历的月大月小、闰年闰月}

农历的大月三十天、小月二十九天是怎样按排的?

我们先说明什么叫朔望月. 出现相同月面所间隔的时间称为朔望月, 也就是从满月 (望) 到下一个满月, 从新月 (朔)到下一个新月, 从蛾眉月 (弦) 到下一个同样的蛾眉月所问隔的时间. 我们把朔望月取作农历月.

已经知道朔望月是 29.5306 天, 把小数部分展为连分数

\[
{0.5306} = \frac{1}{1} \cplus \frac{1}{1} \cplus \frac{1}{7} \cplus \frac{1}{1} \cplus \frac{1}{2} \cplus \frac{1}{33} \cplus \frac{1}{1}\cplus\frac{1}{2} ,
\]

它的渐近分数是

\[
\frac{1}{1},\frac{1}{2},\frac{8}{15},\frac{9}{17},\frac{26}{49},\frac{867}{1634},\frac{893}{1633}
\]

也就是说, 就一个月来说, 最近似的是 30 天, 两个月就应当一大一小, 而 15 个月中应当 8 大 7 小, 17 个月中 9 大 8 小等等. 就 49 个月来说, 前两个 17 个月里, 都有 9 大 8 小, 最后 15 个月里, 有 8 天 7 小, 这样在 49 个月中, 就有 26 个大月.

再谈农历的闰月的算法. 地球绕日一周需 365.2422 天, 朔望月是 29.5306 天, 而它正是我们通用的农历月, 因此一年中应该有

\[
\frac{365.2432}{29.5306} = {12.37}\cdots = {12}\frac{10.8750}{29.5306}.
\]

个农历的月份,也就是多于 12 个月. 因此农历有些年是 12 个月; 而有些年有 13 个月, 称为闰年. 把 0.37 展成连分数

\[
{0.37} = \frac{1}{2} \cplus \frac{1}{1} \cplus \frac{1}{2} \cplus \frac{1}{2} \cplus \frac{1}{1} \cplus \frac{1}{3}
\]

它的渐近分数是

\[
\frac{1}{2},\frac{1}{3},\frac{3}{8},\frac{7}{19},\frac{10}{27}
\]

因此, 两年一闰太多, 三年一闰太少, 八年三闰太多, 十九年七闰太少. 如果算得更精密些

\[
\frac{10.8750}{29.5306} = \frac{1}{2} + \frac{1}{1} \cplus \frac{1}{2} \cplus \frac{1}{1} \cplus \frac{1}{1} \cplus \frac{1}{16} \cplus \frac{1}{1} \cplus \frac{1}{5} \cplus \frac{1}{2} \cplus \frac{1}{6} \cplus \frac{1}{2} \cplus \frac{1}{2}\text{.}
\]

它的渐近分数是

\[
\frac{1}{2},\frac{1}{3}, \frac{3}{8}, \frac{4}{11}, \frac{7}{19}, \frac{116}{315}, \frac{123}{334}, \frac{731}{1985}
\]

\section{火星大冲}

我们知道地球和火星差不多在同一平面上围绕太阳旋转, 火星轨道在地球轨道之外. 当太阳、地球和火星在一直线上并且地球在太阳和火星之间时, 这种现象称为\textbf{冲}. 在冲时地球和火星的距离比冲之前和冲之后的距离都小, 因此便于观察. 地球轨道和火星轨道之间的距离是有远有近的. 在地球轨道和火星轨道最接近处发生的冲叫\textbf{大冲}. 理解冲的现象最方便的办法是看钟面. 时针和分针相重合就是冲. 12小时中有多少次冲? 分针一小时走 \({360}^{ \circ }\left( { = {2\pi }}\right)\) ,时针走 \({30}^{ \circ }\left( { = \frac{2\pi }{12}}\right)\) .

从 12 点正开始,走了 \(t\) 小时后,分针和时针的角度差是

\[
\left( {{2\pi } - \frac{2\pi }{12}}\right) t
\]

如果两针相重,那么这差额应是 \({2\pi }\) 的整数倍,也就是要求出那些 \(t\) 满足下列等式:

\[
\left( {{2\pi } - \frac{2\pi }{12}}\right) t = {2\pi n}.
\]

其中 \(n\) 是整数,也就是要找 \(t\) 使

\[
\frac{11}{12}t
\]

是整数,即在 \(\frac{12}{11},2 \times \frac{12}{11},3 \times \frac{12}{11},\cdots \cdots\) 小时时,分针和时针发生了冲, 在 12 小时中共有 11 次冲.

现在回到火星大冲问题. 火星绕日一周需 687 天, 地球绕日一周需 \({365}\frac{1}{4}\) 天. 把它们的比展成连分数

\[
\frac{687}{365.25} = 1 + \frac{1}{1} \cplus \frac{1}{7} \cplus \frac{1}{2} \cplus \frac{1}{1} \cplus \frac{1}{1} \cplus \frac{1}{1} \cplus \frac{1}{1}
\]

取一个渐近分数

\[
1 + \frac{1}{1} + \frac{1}{7} = \frac{15}{8}
\]

它说明地球绕日 15 圈和火星绕日 8 圈的时间差不多相等, 也就是大约 15 年后火星地球差不多回到了原来的位置, 即从第一次大冲到第二次大冲需间隔 15 年. 上一次大冲在 1956 年 9 月, 下一次约在 1971 年 8 月.

再看看冲的情况如何? 每一天地球转过 \(\frac{2\pi}{365.25}\) 度,火星转过 \(\frac{2\pi}{687}\) 度. 我们看在什么时候太阳、地球和火星在一直线上. 在 \(t\) 天之后,地日火的夹角等于

\[
\left( {\frac{2\pi}{365.25} - \frac{2\pi}{687}}\right) t
\]

如果三者在一直钱上, 并且地球在太阳和火星之间, 那么有整数 \(n\) 使

\[
\left( {\frac{2\pi }{365.25} - \frac{2\pi }{687}}\right) t = {2\pi n}
\]

即

\[
t = \frac{{687} \times {365.25}}{321.75} \times n = {730} \times n,
\]

于是当 \(n = 1,2,\cdots \cdots\) 时所求出的 \(t\) 都是发生冲的时间. 所以约每隔 2 年 50 天有一次冲.

\paragraph*{注意 1}
对于冲的发生可以严格要求三星一钱, 但对于大冲仅要求差不多共线就行了. 然而二者都要求地球在太阳和火星之间.

\paragraph*{注意 2}
如果钟面上还有秒针, 问是否可能三针重合?

\section{日月食}

前面已经介绍过朔望月, 现在再介绍交点月. 大家知道地球绕太阳转, 月亮绕地球转. 地球的轨道在一个平面上, 称为\textbf{黄道面}. 而月亮的轨道并不在这个平面上, 因此月亮轨道和这黄道面有\textbf{交点}. 具体地说, 月亮从地球轨道平面的这一侧穿到另一侧时有一个交点, 再从另一侧又穿回这一侧时又有一个交点, 其中一个在地球轨道圈内, 另一个在圈外. 从圈内交点到圈内交点所需时间称为\textbf{交点月}。 交点月约为 27.2123 天。

当太阳、月亮和地球的中心在一直线上, 这时就发生日食 (如图\ref{fig:1}) 或月食 (如果月亮在地球的另一侧). 如图 \ref{fig:1} , 由于三点在一直线上,因此月亮一定在地球轨道平面上, 也就是月亮在交点上; 同时也是月亮全黑的时候, 也就是朔. 从这样的位置再回到同样的位置必需要有二个条件: 从一交点到同一交点 (这和交点月有关); 从朔到朔(这和朔望月有关). 现在我们来求朔望月和交点月的比.


\begin{figure}[th]
\begin{center}
  \includegraphics[max width=0.4\textwidth]{images/01918226-59b6-7e19-8621-68dc08bb6d69_21_625668.jpg}
  \caption{\label{fig:1}}
\end{center}
\end{figure}

我们有

\[
\frac{29.5306}{27.2123} = 1 + \frac{1}{11} \cplus \frac{1}{1}  \cplus \frac{1}{2}  \cplus \frac{1}{1}  \cplus \frac{1}{4}  \cplus \frac{1}{2}  \cplus \frac{1}{9}  \cplus \frac{1}{1}  \cplus \frac{1}{25}  \cplus \frac{1}{2},
\]

考虑渐近分数

\[
1 + \frac{1}{11}  \cplus \frac{1}{1}  \cplus \frac{1}{2}  \cplus \frac{1}{1}  \cplus \frac{1}{4} = \frac{242}{223}
\]

而 \(\;{223} \times {29.5306}\) 天 \(= {6585}\) 天 \(= {18}\) 年 \(11\) 天,

这就是说, 经过了 \(242\) 个交点月或 $223$ 个朔望月以后, 太阳、 月亮和地球又差不多回到了原来的相对位置. 应当注意的是不一定这三个天体的中心准在一直线上时才出现日食或月食, 稍偏一些也会发生, 因此在这 18 年 11 天中会发生好多次目食和月食 (约有 41 次日食和 29 次月食), 虽然相邻两次日食 (或月食) 的间隔时间并不是一个固定的数, 但是经过了 18 年11 天以后, 由于这三个天体又回到了原来的相对位置, 因此在这 18 年 11 天中日食、月食发生的规律又重复实现了。这个交食 (日食月食的总称) 的周期称为沙罗周期. “沙罗”就是重复的意思\footnote{来自拉丁语 Saros(校对者注)}. 求出了沙罗周期, 就大大便于日食月食的测定.

\section{日月合璧,五星联珠,七曜同宫}

今年(1962年)二月五日那天,正当我们欢度春节的时候, 天空中出现了一个非常罕见的现象, 那就是金、木、水、火、土五大行星在同一方向上出现, 而且就在这方向上日食也正好发生. 这种现象称为\textbf{日月合璧, 五星联珠, 七曜同宫}(图 \ref{fig:2}), 这是几百年才出现一次的现象.

\begin{figure}[th]
\begin{center}
  \includegraphics[max width=0.9\textwidth]{images/01918226-59b6-7e19-8621-68dc08bb6d69_22_592035.jpg}
  \caption{\label{fig:2}}
\end{center}
\end{figure}

天文学家把“天”划分成若干部分, 每一部分称为一个星座. 通过黄道面的共有 12 个星座, 称为\textbf{黄道十二宫}. 这次金、木、水、火、土、日、月七个星球同时走到了一个宫内(宝瓶宫), 而日食也在这宫内发生.

现在我们根据下表来说明这种现象是怎样发生的.

\begin{center}
\adjustbox{max width=\textwidth}{
\begin{tabular}{|c|c|c|c|c|c|c|c|}
\hline
星 别 & 水 星 & 金 星 & 火 星 & 土 星 & 木 星 & 太 阳 & 月 亮\\
\hline
赤 经 & \({318}^{ \circ }{15}^{\prime }\) & \({320}^{ \circ }{30}^{\prime }\) & \({319}^{ \circ }{45}^{\prime }\) & \({321}^{ \circ }{15}^{\prime }\) & \({323}^{ \circ }{45}^{\prime }\) & \({318}^{ \circ }{15}^{\prime }\) & $818^{\circ}$ \\
\hline
赤 纬 & \({12}^{ \circ }{24}^{\prime }\) & \({16}^{ \circ }{45}^{\prime }\) & \({20}^{ \circ }{36}^{\prime }\) & \({19}^{ \circ }{40}^{\prime }\) & \({15}^{ \circ }{54}^{\prime }\) & \({16}^{ \circ }{08}^{\prime }\) & \({15}^{ \circ }{57}^{\prime }\) \\
\hline
\end{tabular}
}
\end{center}

表中的赤经和赤纬表示某一星球的方向. 如果两个星球对应的赤经和赤纬很接近, 那么在地球上看起来, 它们在同一个方向上出现. 表中所列是 1962 年二月五日那天各星球的方向, 由此可见它们方向的相差是不大的. 怎样来理解不大? 钟面上每一小时代表 \({360}^{ \circ }/{12} = {30}^{ \circ }\) ,每一分钟代表 \({30}^{ \circ }/5 = {6}^{ \circ }\) ,也就是一分钟的角度是 \({6}^{ \circ }\) . 这就可以看出这七个星球的方向是多么互相接近了.

为什么又称为五星联珠呢? 我们看起来, 那天金、木、水、 火、土五星的位置差不多在一起, 但实际上它们是有远有近的, 因此好象串成了一串珠子一样. 这种现象也称为五星聚. 古代迷信的人把五星联珠看作吉祥之兆, 因此把相差不超过 \({45}^{ \circ }\) 的情况都称为五星联珠了.

关于这种现象, 远在二千多年前, 我国历史上就有了记载. 在《汉书》律历志上是这样写的:
\begin{quotation}
  \textit{复复太初历, 晦朔弦望皆最窘, 日月如舍璧, 五星如连珠}
  \end{quotation}
  而且还有一个注,
\begin{quotation}
\textit{太初上元甲子夜半朔旦各至时, 七曜皆会聚斗牵牛分度。 夜足如何壁连珠。}
\end{quotation}

太初是汉武帝的年号, 在公元前 104 年.

读者一定希望知道何时再发生几个星球的联珠现象, 我们在下面两节中提出一个考虑这个问题的粗略方法.

\section{计算方法}

我们用以下方法解决类似于上节所提出的问题.

\paragraph*{问题 1} 假定有内外两圈圆跑道, 甲在里圈沿反时针方向匀速行走, 49 分走完一围;乙在外圈也沿反时针方向匀速行走, 86 分钟走完一圈. 出发时他们和圆心在一直线上. 问何时甲、乙在圆心所张的角度小于 \({15}^{ \circ }\) ?

\begin{figure}[h]
\begin{center}
  \includegraphics[max width=0.5\textwidth]{images/01918226-59b6-7e19-8621-68dc08bb6d69_24_357475.jpg}
  \caption{\label{fig:3}}
\end{center}
\end{figure}


解 甲每分钟行走 \(\frac{2\pi}{49}\) 度,乙每分钟行走 \(\frac{2\pi }{86}\) 度. \(t\) 分钟后,走过的角度差是 \(\left( {\frac{2\pi}{49} - \frac{2\pi }{86}}\right) t\) . 所以他们与圆心连结的角度是

\[
\theta = \left( {\frac{2x}{49} - \frac{2x}{86}}\right) t - {2\pi m}
\]

这儿 \(m\) 是一个自然数,使 \(\theta\) 的绝对值最小. 问题一变而为 \(t\) 是何值时,存在自然数 \(m\) 使

\[
\left| {\left( {\frac{2\pi }{49} - \frac{2\pi }{86}}\right) t - {2\pi m}}\right| < {15}^{ \circ } = \frac{{2\pi } \times {15}}{860} = \frac{2\pi }{24},
\]

也就是

\[
\left| {\frac{37t}{4214} - m}\right| < \frac{1}{24}
\]

取 \(m = 0\) ,得

\[
t < \frac{4214}{{37} \times {24}} = {4.75}
\]

即在出发后 4.75 分钟之内夹角都小于 \({15}^{ \circ }\) .

取 \(m = 1\) ,得

\[
\frac{23}{24} < \frac{37}{4214}t < \frac{25}{24}
\]

即

\[
{109.15} \leq t \leq {118.64}
\]

也就是说,出发 4.75 分钟后,夹角开始变得大于 \({15}^{ \circ }\) ; 在出发后的 109.15 分到 118.64 分之间时,夹角又在 \({15}^{ \circ }\) 内.

一般地讲,

\[
m - \frac{1}{24} < \frac{37}{4214}t < m + \frac{1}{24}
\]

\[
\text{即}\;\frac{4214}{37}m - \frac{4214}{2 \cdot 1 \times {37}} < t < \frac{4214}{37}m + \frac{4214}{{37} \times {24}}
\]

\begin{equation}
\label{eq:4}
{113.9m} - {4.75} < t < {113.9m} + {4.75}
\end{equation}

这个问题看来较难, 而实质上比本文所讨论的其它问题都更容易. 把这问题代数化一下: 假定甲、乙各以 \(a,b\) 分钟走完一圈, 那么

\[
\left| {\left( {\frac{1}{a} - \frac{1}{b}}\right) t - m}\right| \leq \frac{1}{24}
\]

即 \(\frac{ab}{b - a}\left( {m - \frac{1}{24}}\right) < t < \frac{ab}{b - a}\left( {m + \frac{1}{24}}\right) ,m = 1,2,3,\cdots\) .

\paragraph*{问题 2} 如果还有一圈, 丙以 180 分钟走完一圈, 问何时三个人同在一个 \({15}^{ \circ }\) 的角内?

解 在直线上用红铅笔标上区间 \ref{eq:4}, 即从 0 到 4.75, 113.9-4.75 到 \({113.9} + {4.75},{113.9} \times 2 - {4.75}\) 到 \({113.9} \times 2 + {4.75},\cdots \cdots\) 分别涂上 红色. 这是甲、乙同在 \({15}^{ \circ }\) 角内的时间. 同法用绿色线标出甲、丙同在 \({15}^{ \circ }\) 角内的时间,用蓝色线标出乙、丙同在 \({15}^{ \circ }\) 角内的时间. 那么三色线段的重复部分就是甲、乙、丙三人同在 \({15}^{ \circ }\) 角内的时间.

当然, 只是为了方便, 才用各色线来标出结果, 读者还是应当把它具体地计算出来.

现在我们回到第十节中所提出的问题, 但把问题设想得简单一点. 假设各行星在同一平面上, 以等角速度绕太阳旋转, 它们绕日一周所需时间列于下表:

\begin{center}
\adjustbox{max width=\textwidth}{
\begin{tabular}{|c|c|c|c|c|c|c|}
\hline
星 别 & 水星 & 金 星 & 地球 & 火 星 & 木 星 & 土 星 \\
\hline
绕日周期 & 88天 & 225天 & 1 年 \(= {365}\) 天 & 1 年322日 & 11年315日 & 29年167日 \\
\hline
\end{tabular}
}
\end{center}

假设在 1962 年二月五日, 地球、金、木、水、火、土等星球, 位于以太阳为中心的圆的同一半径上, 问经过多少时间以后它们都在同一个 \({30}^{ \circ }\) 的圆心角内?

这个问题可以用上面所介绍的方法解决. 当然, 得到的结果是很粗略的, 原因是各行星并非在一个平面上运动, 而且它们也不是作等角速度运动, 所以实际情况很复杂. 但读者不妨作为练习照上面的方法去计算一下.

\section{有理数逼近实数}

以上所讲的一些问题, 可以概括并推广如下:

始定实数 \(\alpha \left( { > 0}\right)\) ,要求找一个有理数 \(\frac{p}{q}\) 去逼近它,说得更确切些,给一自然数 \(N\) ,找一个分母不大于 \(N\) 的有理数 \(\frac{p}{q}\) ,

使误差

\[
\left| {\alpha - \frac{p}{q}}\right|
\]

最小.

这是一个重要问题. 由它引导出数论的一个称为丢番图 (Diophantine) 逼近论的分支. 它也可以看成数学上各种各样逼近论的开端.

以上所讲的感性知识告诉我们,如果 \(\alpha\) 是一有理数,我们把 \(\alpha\) 展开成速分数,而命 \(\frac{{p}_{n}}{{q}_{n}}\) 为其第 \(n\) 个渐近分数,那么在分母不大于 \({q}_{n}\) 的一切分数中,以 \(\frac{{p}_{n}}{{q}_{n}}\) 和 \(\alpha\) 最为接近. 我们将在第十五节中证明这一事实. 不但如此,这个事实对于 \(\alpha\) 是无理数的情形也同样正确. 为此, 我们需要介绍把无理数展成连分数的方法.

在第三节中,我们已用辗转相除法把一有理数 \(\frac{a}{b}\) 展成连分数. 现在把那里的 \ref{eq:1},\ref{eq:2} 诸式加以改写,使得

\[
\left\{ \begin{array}{ll} \dfrac{a}{b} = {a}_{0} + \dfrac{r}{b} & \left( {0 < r < b}\right) \\ \dfrac{b}{r} = {a}_{1} + \dfrac{{r}_{1}}{r} & \left( {0 < {r}_{1} < r}\right) \\ \cdots \cdots \cdots \cdots & \\ \dfrac{{r}_{n - 3}}{{r}_{n - 2}} = {a}_{n - 1} + \dfrac{{r}_{n - 1}}{{r}_{n - 2}} & \left( {0 < {r}_{n - 1} < {r}_{n - 2}}\right) \\ \dfrac{{r}_{n - 2}}{{r}_{n - 1}} = {a}_{n} & \end{array}\right.
\]

我们看到: \({a}_{0},{a}_{1},\cdots ,{a}_{n - 1},{a}_{n}\) 实际上也就是用 \(b\) 除 \(a\) ,用 \(r\) 除 \(b,\cdots \cdots\) ,用 \({r}_{n - 1}\) 除 \({r}_{n - 2}\) 以及用 \({r}_{n}\) 除 \({r}_{n - 1}\) 后所得各个商数的整数部分. 如果以配号 \(\left\lbrack x\right\rbrack\) 来表示实数 \(x\) 的整数都分 (即不大于 \(x\) 的最大整数,例如 \(\left\lbrack 2\right\rbrack = 2,\left\lbrack \pi \right\rbrack = 3,\left\lbrack {-{1.5}}\right\rbrack = - 2\) 等), 那末

\[
{a}_{0} = \left\lbrack \frac{a}{b}\right\rbrack ,\;{a}_{1} = \left\lbrack \frac{b}{r}\right\rbrack ,\cdots ,{a}_{n - 1} = \left\lbrack \frac{{r}_{n - 3}}{{r}_{n - 2}}\right\rbrack ,\;{a}_{n} = \left\lbrack \frac{{r}_{n - 2}}{{r}_{n - 1}}\right\rbrack ,
\]

而 \(\frac{a}{b}\) 就有如下的连分数表示:

\[
\frac{a}{b} = {a}_{0} + \frac{1}{{a}_{1}} \cplus \frac{1}{{a}_{2}} \cplus \cdots \cplus \frac{1}{{a}_{n - 1}} \cplus \frac{1}{{a}_{n}}
\]

对于无理数 \(\alpha\) ,我们也可以用这方法将它以连分数表示. 首先取 \(a\) 的整数部分 \(\left\lbrack \alpha \right\rbrack\) ,用 \({a}_{0}\) 记之,然后看 \(\alpha\) 和 \({a}_{0}\) 的差, \(\alpha - {a}_{0} = \frac{1}{{\alpha }_{1}}\) (注意,因为 \(\alpha\) 是无理数, \({\alpha }_{1}\) 一定大于 1); 再取 \({\alpha }_{1}\) 的整数部分 \(\left\lbrack {\alpha }_{1}\right\rbrack\) ,记它为 \({ a}_{1}\) ,而改写 \({\alpha }_{1}\) 和 \({a}_{1}\) 的差, \({\alpha }_{1} - {\alpha }_{1} = \frac{1}{{\alpha }_{2}}\) (注意, \({\alpha}_{2} > 1\) ); 再取 \({\alpha}_{2}\) 的整数部分为 \({a}_{2}\cdots \cdots\) 等等. 也就是说, 命

\[
{a}_{0} = \left\lbrack \alpha \right\rbrack ,\;\alpha - {a}_{0} = \frac{1}{{\alpha }_{1}},
\]

\[
{a}_{1} = \left\lbrack {\alpha }_{1}\right\rbrack ,\;{\alpha }_{1} - {a}_{1} = \frac{1}{{\alpha }_{2}},
\]

\[
{a}_{2} = \left\lbrack {\alpha }_{2}\right\rbrack ,\;{\alpha}_{2} - {a}_{2} = \frac{1}{{\alpha }_{3}},
\]

........................

于是显然有\footnote{有理数 $\frac{a}{b}$ 的连分数表示一定是有尽的,而无理数 $\alpha$ 的连分数表示则一定无尽。(原注)}

\[
a = {a}_{0} + \frac{1}{{a}_{1}} = {a}_{0} + \cfrac{1}{{a}_{1} + \cfrac{1}{{a}_{2}}} = \cdots = {a}_{0} + \cfrac{1}{{a}_{1} + \cfrac{1}{{a}_{2} + \cfrac{1}{{a}_{3} + \cdots }}}
\]


\[= {a}_{0} + \frac{1}{{a}_{1}} \cplus \frac{1}{{a}_{2}} \cplus \frac{1}{{a}_{3}} \cplus \cdots \cplus \frac{1}{{a}_{n}} + \cdots \]



在第五节开头我们就是按照这个方法去求 \(\pi\) 的连分数的. 和有理数的情形一样, 称

\[
{a}_{0} + \frac{1}{{a}_{1}} \cplus \frac{1}{{a}_{2}}  \cplus \cdots  \cplus \frac{1}{{a}_{n}}
\]

为 \(\alpha\) 的第 \(n\) 个渐近分数. 关于渐近分数的一些基本性质,将在下节中加以说明.

上面已经说过,我们将在第十五节中证明: 如果命 \(N = {q}_{n}\) , 则 \(\frac{{p}_{n}}{{q}_{n}}\) 的确是使 \(\left| {\alpha - \frac{p}{q}}\right| \left( {q \leq N = {q}_{n}}\right)\) 为最小的有理数.

但是井非仅有 \(\frac{{p}_{n}}{{q}_{n}}\) 有这种性质,例如,在第三节中,我们已经给出例子:

\[
\alpha = \frac{543}{236},N = 7,
\]

而 \(\frac{p}{q} = \frac{16}{7}\) 在所有分母不大于 7 的分数中最接近于 \(\alpha\) ,但 \(\frac{16}{7}\) 并非 \(\frac{543}{236}\) 的渐近分数.

\section{渐近分数}

设 \(\alpha\) 是一正数,并且假定它已展成连分数

\[
\alpha = {a}_{0} + \frac{1}{{a}_{1}} \cplus \frac{1}{{a}_{2}} \cplus \cdots
\]

容易看到, 它的前三个渐近分数是

\[
\frac{{a}_{0}}{1},\frac{{a}_{1}{a}_{0} + 1}{{a}_{1}},\frac{{a}_{2}\left( {{a}_{1}{a}_{0} + 1}\right) + {a}_{0}}{{a}_{2}{a}_{0} + 1}.
\]

一般地, 有
\begin{theorem}
  \label{thm:1}
  如命
\[
{p}_{0} = {a}_{0},\;{p}_{1} = {a}_{1}{a}_{0} + 1,\;{p}_{n} = {a}_{n}{p}_{n - 1} + {p}_{n - 1}\;\left( {n \geq 2}\right) ,
\]

\[
{q}_{0} = 1,\;{q}_{1} = {a}_{1},\;{q}_{n} = {a}_{n}{q}_{n - 1} + {q}_{n - 1}\;\left( {n \geq 2}\right) ,
\]

那么 \(\frac{{p}_{n}}{{q}_{n}}\) 就是 \(\alpha\) 的第 \(n\) 个渐近分数.
\end{theorem}

\begin{proof}
  当 \(n = 2\) 时,定理已经正确. 现在用数学归纳法证明定理.

我们看到, \(\alpha\) 的第 \(n - 1\) 个渐近分数

\[
{a}_{0} + \frac{1}{{a}_{1}} \cplus \frac{1}{{a}_{2}} \cplus \cdots \cplus \frac{1}{{a}_{n - 1}}
\]

和 \(\alpha\) 的第 \(n\) 个渐近分数

\[
{a}_{0} + \frac{1}{{a}_{1}} \cplus \frac{1}{{a}_{2}} \cplus \cdots \cplus \frac{1}{{a}_{n - 1}} \cplus \frac{1}{{a}_{n}}
\]

的差别仅在于将 \({a}_{n - 1}\) 换成 \({a}_{n - 1} + \frac{1}{{a}_{n}}\) . 所以若定理对 \(n - 1\) 正确,也就是如果 \(\alpha\) 的第 \(n - 1\) 个渐近分数是

\[
\frac{{p}_{n - 1}}{{q}_{n - 1}} = \frac{{a}_{n - 1}{p}_{n - 2} + {p}_{n - 3}}{{a}_{n - 1}{q}_{n - 2} + {q}_{n - 3}}
\]

那么第 \(n\) 个渐近分数应是
\begin{align*}
  &\frac{\left( {{a}_{n - 1} + \frac{1}{{a}_{n}}}\right) {p}_{n - 2} + {p}_{n - 3}}{\left( {{a}_{n - 1} + \frac{1}{{a}_{n}}}\right) {q}_{n - 2} + {q}_{n - 3}} \\
  = & \frac{{a}_{n}\left( {{a}_{n - 1}{p}_{n - 2} + {p}_{n - 3}}\right) + {p}_{n - 2}}{{a}_{n}\left( {{a}_{n - 1}{q}_{n - 2} + {q}_{n - 3}}\right) + {q}_{n - 3}}\\
= & \frac{{a}_{n}{p}_{n - 1} + {p}_{n - 2}}{{a}_{n}{q}_{n - 1} + {q}_{n - 2}} = \frac{{p}_{n}}{{q}_{n}}
\end{align*}


定理得到证明.
\end{proof}


有了这个递推公式,我们就可以根据 \(\alpha\) 的连分数立刻写出它的各个渐近分数.

如果命

\[
{\alpha}_{n} = {a}_{n} + \frac{1}{{a}_{n + 1}} \cplus \frac{1}{{a}_{n + 2}} + \cdots ,
\]

那么显见

\[
\alpha = {a}_{0} + \frac{1}{{a}_{1}} \cplus \frac{1}{{a}_{2}} \cplus \cdots \cplus \frac{1}{{a}_{n - 1}} \cplus \frac{1}{{\alpha}_{n}}.
\]

它和 \(\alpha\) 的第 \(n\) 个渐近分数的差别仅在于将 \({a}_{n}\) 换成 \({\alpha }_{n}\) ,于是由定理 1 立刻得到

\begin{theorem}
  \label{thm:2}
\[
\alpha = {\alpha }_{0},\;\alpha = \frac{{\alpha }_{1}{\alpha }_{0} + 1}{{\alpha }_{1}}.\;\alpha = \frac{{\alpha }_{n}{p}_{n - 1} + {p}_{n - 2}}{{\alpha }_{n}{q}_{n - 1} + {q}_{n - 2}}\;\left( {n \geq 2}\right) .
\]
\end{theorem}

\begin{theorem}
  \label{thm:3}
\[
{p}_{n}{q}_{n - 1} - {q}_{n}{p}_{n - 1} = {\left( -1\right) }^{n - 1},\;\left( {n \geq 1}\right) ,
\]

\[
{p}_{n}{q}_{n - 2} - {q}_{n}{p}_{n - 2} = {\left( -1\right) }^{n}{a}_{n},\;\left( {n \geq 2}\right) .
\]
\end{theorem}

\begin{proof}
  易见
\[
{p}_{1}{q}_{0} - {q}_{1}{p}_{0} = \left( {{a}_{0}{a}_{1} + 1}\right) - {a}_{1}{a}_{0} = 1.
\]

由定理 \ref{thm:1} 可知
\begin{align*}
  &{p}_{n}{q}_{n - 1} - {q}_{n}{p}_{n - 1}\\
  =& \left( {{a}_{n}{p}_{n - 1} + {p}_{n - 2}}\right) {q}_{n - 1} - \left( {{a}_{n}{q}_{n - 1} + {q}_{n - 2}}\right) {p}_{n - 1}\\
=& - \left( {{p}_{n - 1}{q}_{n - 2} - {q}_{n - 1}{p}_{n - 2}}\right) .
\end{align*}
故由数学归纳法, 立刻得出第一个式子.

仍用定理 \ref{thm:1} 和第一式, 得出
\begin{align*}
  &{p}_{n}{q}_{n - 2} - {q}_{n}{p}_{n - 2} \\
  = & \left( {{a}_{n}{p}_{n - 1} + {p}_{n - 2}}\right) {q}_{n - 2} - \left( {{a}_{n}{q}_{n - 1} + {q}_{n - 2}}\right) {p}_{n - 2}\\
  = & {\left( -1\right) }^{n}{a}_{n}\text{.}
\end{align*}
\end{proof}

从定理 \ref{thm:3} 的第一式可以看到, \({p}_n\) 与 \({q}_n\) 的任何公约数,一定除得尽 \({\left( -1\right) }^{n - 1}\) ,所以得到

\begin{corollary}
  \({p}_{n}\) 和 \({q}_{n}\) 互素 (即它们的最大公约数是 1 ).
\end{corollary}

\begin{theorem}
  \label{thm:4}
  \[\;\alpha - \frac{{p}_{n}}{{q}_{n}} = \frac{{\left( -1\right) }^{n}}{{q}_{n}\left( {{\alpha }_{n + 1}{q}_{n} + {q}_{n - 1}}\right) } = \frac{{\left( -1\right) }^{n}{\alpha }_{n + 2}}{{q}_{n}\left( {{\alpha }_{n + 2}{q}_{n + 1} + {q}_{n}}\right) }\]
\end{theorem}

\begin{proof}
  由定理 \ref{thm:2} 及定理 \ref{thm:3},

\[
\alpha - \frac{{p}_{n}}{{q}_{n}} = \frac{{\alpha }_{n + 1}{p}_{n} + {p}_{n - 1}}{{\alpha }_{n + 1}{q}_{n} + {q}_{n - 1}} - \frac{{p}_{n}}{{q}_{n}} = \frac{{\left( -1\right) }^{n}}{{q}_{n}\left( {{\alpha }_{n + 1}{q}_{n} + {q}_{n - 1}}\right) }.
\]

和

\[
\alpha - \frac{{p}_{n}}{{q}_{n}} = \frac{{\alpha }_{n + 2}{p}_{n + 1} + {p}_{n}}{{\alpha }_{n + 2}{q}_{n + 1} + {q}_{n}} - \frac{{p}_{n}}{{q}_{n}} = \frac{{\left( -1\right) }^{n}{\alpha }_{n + 1}}{{q}_{n}\left( {{\alpha }_{n + 2}{q}_{n + 1} + {q}_{n}}\right) }.
\]
\end{proof}


\section{实数作为有理数的极限}

在本节中,我们假定 \(\alpha\) 是无理数. 由上节定理 3 推得

\[
\frac{{p}_{n}}{{q}_{n}} - \frac{{p}_{n - 1}}{{q}_{n - 1}} = - \frac{{\left( -1\right) }^{n - 1}}{{q}_{n}{q}_{n - 1}}
\]

\[
\frac{{p}_{n}}{{q}_{n}} - \frac{{p}_{n - 2}}{{q}_{n - 2}} = \frac{{\left( -1\right) }^{n}{a}_{n}}{{q}_{n}{q}_{n - 2}}
\]

由此并由定理 \ref{thm:4}, 我们得到

\[
\frac{{p}_{0}}{{q}_{0}} < \frac{{p}_{2}}{{q}_{2}} < \frac{{p}_{4}}{{q}_{4}} < \cdots < \frac{{p}_{2n}}{{q}_{2n}} < \cdots < \alpha.
\]

和

\[
\frac{{p}_{1}}{{q}_{1}} > \frac{{p}_{3}}{{q}_{3}} > \frac{{p}_{5}}{{q}_{5}} > \cdots > \frac{{p}_{{2n} + 1}}{{q}_{{2n} + 1}} > \cdots > \alpha ,
\]

而且

\[
\left| {\frac{{p}_{2n}}{{q}_{2n}}\cdots \frac{{p}_{{2n} - 1}}{{q}_{{2n} - 1}}}\right| = \frac{1}{{q}_{2n}{q}_{{2n} - 1}}
\]

当 \(n\) 无限增大时,由上节定理 \(1,{q}_{n} = {a}_{n}{q}_{n - 1} + {q}_{n - 2} > {q}_{n - 1}\) . 因为 \({q}_{1} = 1\) ,所以 \({q}_{n} \geq n\) ,因此 \({q}_{n}\) 也无限增大. 而 \(\frac{{p}_{2n}}{{\dot{q}}_{2n}}\) 是一递增的数列,趋于极限 \(\alpha ;\frac{{p}_{{2n} + 1}}{{q}_{{2n} + 1}}\) 是一递减的数列,趋于极限 \(\alpha\) (由上节定理 4 可见当 \(n \rightarrow \infty\) 时, \(\left| {a - \frac{{p}_{n}}{{q}_{n}}}\right| \leq \frac{1}{{q}_n^2} \rightarrow 0\) ).

\begin{theorem}
  \label{thm:5}
  \(\frac{{p}_{n}}{{q}_{n}}\) 趋于 \(\alpha\) ,而 \(\frac{{p}_{n}}{{q}_{n}}\) 比 \(\frac{{p}_{n - 1}}{{q}_{n - 1}}\) 更接近于 \(\alpha\) . 也就是

\[
\left| {a - \frac{{p}_{n}}{{q}_{n}}}\right| < \left| {\alpha - \frac{{p}_{n - 1}}{{q}_{n - 1}}}\right|
\]
\end{theorem}

\begin{proof}
  证 由定理 \ref{thm:4} 已知
\[
\alpha - \frac{{p}_{n}}{{q}_{n}} = \frac{{\left( -1\right) }^{n}}{{q}_{n}\left( {{\alpha }_{n + 1}{q}_{n} + {q}_{n - 1}}\right) }
\]

及

\[
\alpha - \frac{{p}_{n - 1}}{{q}_{n - 1}} = \frac{{\alpha }_{n + 1}{\left( -1\right) }^{n - 1}}{{q}_{n - 1}\left( {{\alpha }_{n + 1}{q}_{n} + {q}_{n - 1}}\right) }
\]

由于 \({\alpha }_{n + 1} \geq 1\) 及 \({q}_{n - 1} < {q}_{n}\) ,所以

\[
\frac{1}{{q}_{n}\left( {{\alpha }_{n + 1}{q}_{n} + {q}_{n - 1}}\right) } < \frac{{\alpha }_{n + 1}}{{q}_{n - 1}\left( {{\alpha }_{n + 1}{q}_{n} + {q}_{n - 1}}\right) }
\]

得

\[
\left| {\alpha - \frac{{p}_{n}}{{q}_{n}}}\right| < \left| {\alpha - \frac{{p}_{n - 1}}{{q}_{n - 1}}}\right|
\]

\end{proof}

这证明也给出了

\begin{theorem}
  \label{thm:6}
  \[\frac{1}{{q}_{n - 1}\left( {{q}_{n} + {q}_{n - 1}}\right) } \leq \left| {a - \frac{{p}_{n - 1}}{{q}_{n - 1}}}\right| \leq \frac{1}{{q}_{n - 1}{q}_{n}}\]
\end{theorem}

因此推出

\begin{theorem}
  \label{thm:7}
  有无限多对整数 \(p\text{、}q\) 使
\[
\left| {\alpha - \frac{p}{q}}\right| < \frac{1}{{q}^{2}}
\]
\end{theorem}


\section{最佳逼近}

\paragraph*{问题} 求出所有的 \(\frac{P}{Q}\) ,使它比分母不大于 \(Q\) 的一切分数 (不等于 \(\frac{P}{Q}\) ) 都更接近于 \(\alpha\) ,即要求:

\begin{equation}
\label{eq:6}
\left| {\alpha - \frac{P}{Q}}\right| < \left| {\alpha - \frac{p}{q}}\right| \left( {q \leq Q,\frac{p}{q} \neq \frac{p}{q}}\right) . \tag{1}
\end{equation}

先证一初步结果.

\begin{theorem}
  \label{thm:8}
  当 \(n \geq 1\text{、}q \leq {q}_{n},\frac{p}{q} \neq \frac{{p}_{n}}{{q}_{n}}\) ,那么渐近分数 \(\frac{{p}_{n}}{{q}_{n}}\) 比 \(\frac{p}{q}\) 更接近于 \(\alpha\) 。
\end{theorem}

\begin{proof}
  不妨假设 \(n\) 是偶数,至于 \(n\) 是奇数的 情形可以完全同样地证明.

若 \(\alpha = \frac{{p}_{n}}{{q}_{n}}\) ,定理自然成立. 现在假设 \(\alpha \ne \frac{{p}_{n}}{{q}_{n}}\) ,若 \(\frac{p}{q}\) 比 \(\frac{{p}_{n}}{{q}_{n}}\) 更接近于 \(\alpha\) . 由定理 \ref{thm::5} 可知

\[
\left| {\alpha -\frac{p}{q}}\right| < \alpha - \frac{{p}_{n}}{{q}_{n}} < \frac{{p}_{n - 1}}{{q}_{n - 1}}-\alpha .
\]

即

\[
\alpha - \frac{{p}_{n - 1}}{{q}_{n - 1}} < \alpha - \frac{p}{q} < \alpha - \frac{{p}_{n}}{{q}_{n}}
\]

也就是
\begin{equation}
\label{eq:7}
\frac{{p}_{n}}{{q}_{n}} < \frac{p}{q} < \frac{{p}_{n - 1}}{{q}_{n - 1}} \tag{2}
\end{equation}

所以我们只须证明适合上式的分数 \(\frac{p}{q}\) ,必有分母 \(q > {q}_{n}\) .

如果

\[
\alpha < \frac{p}{q} < \frac{{p}_{n - 1}}{{q}_{n - 1}}
\]

那么

\[
\frac{1}{q{q}_{n - 1}} \leq \frac{{p}_{n - 1}}{{q}_{n - 1}} - \frac{p}{q} < \frac{{p}_{n - 1}}{{q}_{n - 1}} - \alpha = \frac{1}{{q}_{n - 1}\left( {{\alpha}_{n}{q}_{n - 1} + {q}_{n - 2}}\right) },
\]

因此

\[
q > {\alpha }_{n}{q}_{n - 1} + {q}_{n - 2} \geq {\alpha}_{n}{q}_{n - 1} + {q}_{n - 2} = {q}_{n}.
\]

同样地由

\[
\frac{{p}_{n}}{{q}_{n}} < \frac{p}{q} < \alpha
\]

可以得出 \(q > {q}_{n + 1} > {q}_{n}\) . 于是定理得到证明.
\end{proof}


在定理的证明过程中, 我们还推出下述的渝断,

\begin{corollary}
  \label{cor:8.1}
  若 \(\frac{p}{q}\) 在 \(\frac{{p}_{n}}{{q}_{n}}\) 和 \(\alpha\) 之间、那就必有 \(q > {q}_{n + 1}\) .
\end{corollary}



定理 \ref{thm:8} 说明渐近分数满足本节开始所提问题中对 \(\frac{P}{Q}\) 的要求 ( 1 ),但我们还不知道能满足 ( 1 ) 的 \(\frac{P}{Q}\) 是否仅限于渐近分数. 关于这个问题, 我们有下面的定理.

\begin{theorem}
  \label{thm:9}
  \begin{enumerate}[label=\textup{(\roman*)}]
  \item\label{item:9.1} 在分母不大于 \({q}_{1} = {a}_{1}\) 的一切分数中,只有

\[
{a}_{0} + \frac{1}{a}\qquad\left( {\frac{{a}_{1} + 1}{2} \leq q \leq {a}_{1}}\right)
\]

满足\eqref{eq:6}.
\item\label{item:9.2} 设 \(n \geq 2\) . 在分母大于 \({q}_{n - 1}\) 、但不大于 \({q}_{n}\) 的一切分数中. 只有

\[
\frac{i{p}_{n - 1} + {p}_{n - 2}}{i{q}_{n - 1} + {q}_{n - 2}}\qquad\left( {\frac{1}{2}\left( {{a}_{n} - \frac{{q}_{n - 2}}{{q}_{n - 1}}}\right) < i \leq {a}_{n}}\right)
\]

满足\eqref{eq:6}.
  \end{enumerate}
\end{theorem}

\begin{proof}
  先证\ref{item:9.1}. 我们有

\[
{a}_{0} < \alpha \leq {a}_{0} + \frac{1}{\alpha} \leq {\alpha}_{0} + \frac{1}{q}\quad\left( {q \leq {a}_{1}}\right) ,
\]

\({a}_{0}\) 和 \({a}_{0} + \frac{1}{{q}_{1}}\) 至 \(\alpha\) 的距离分别等于 \(\frac{1}{{\alpha }_{1}}\) 和 \(\frac{1}{q} - \frac{1}{{\alpha }_{1}}\) ,所以当而且仅当 \({2q} > {a}_{1}\) ,或即 \(q \geq \frac{{a}_{1} + 1}{2}\) 时, \({a}_{0} + \frac{1}{q}\) 才比 \({a}_{0}\) 更接近于 \(\alpha\) . 又对于任何 \(q,{a}_{0}\) 和 \({a}_{0} + \frac{1}{q}\) 的距离等于 \(\frac{1}{q}\) ,所以 \({a}_{0} + \frac{1}{q}\left( {\frac{{a}_{1} + 1}{2} \leq q}\right.\) \(\left. { \leq {a}_{1}}\right)\) 比分母不大于 \(q\) 的其它任何分数都更接近于 \(\alpha\) .

\ref{item:9.2} 的证明: 我们假设 \(n\) 是偶数, \(n\) 是奇数的情形可以同样证明.

由定理 \ref{thm:3},\ref{thm:4},\ref{thm:5}可知
\begin{equation}
\label{eq:8}
\frac{{p}_{n - 2}}{{q}_{n - 2}} < {2\alpha } - \frac{{p}_{n - 1}}{{q}_{n - 1}} < \frac{{p}_{n}}{{q}_{n}} \leq \alpha < \frac{{p}_{n - 1}}{{q}_{n - 1}}
\end{equation}

(因为 \({2\alpha } - \frac{{p}_{n - 1}}{{q}_{n - 1}}\) 和 \(\frac{{p}_{n - 1}}{{q}_{n - 1}}\) 到 \(\alpha\) 的距离相等,而 \(\frac{{p}_{n - 1}}{{q}_{n - 1}}\) 比 \(\frac{{p}_{n - 2}}{{q}_{n - 2}}\) 更接近于 \(\alpha ,\frac{{p}_{n}}{{q}_{n}}\) 又比 \(\frac{{p}_{n - 1}}{{q}_{n - 1}}\) 更接近于 \(\alpha\) .)

设 \(\frac{P}{Q}\) 满足 \eqref{eq:6} 和 \({q}_{n - 1} < Q \leq {q}_{n}\) ,那么必有

\[
\left| {\frac{P}{Q} - \alpha}\right| < \frac{{p}_{n - 1}}{{q}_{n - 1}} - \alpha
\]

即

\[
\alpha - \frac{{p}_{n - 1}}{{q}_{n - 1}} < \frac{P}{Q} - \alpha < \frac{{p}_{n - 1}}{{q}_{n - 1}} - \alpha ,
\]

也就是

\[
{2a} - \frac{{p}_{n - 1}}{{q}_{n - 1}} < \frac{P}{Q} < \frac{{p}_{n - 1}}{{q}_{n - 1}}
\]

其次,由 \(Q \leq {q}_{n}\) 和系\ref{cor:8.1}. 不可能有 \(\frac{{p}_{n}}{{q}_{n}} < \frac{P}{Q} < \alpha\) ,也不可能有 \(a \leq \frac{P}{Q} < \frac{{p}_{n - 1}}{{q}_{n - 1}}\) . 所以必有

\begin{equation}
\label{eq:9}
{2\alpha } - \frac{{p}_{n - 1}}{{q}_{n - 1}} < \frac{P}{Q} \leq \frac{{p}_{n}}{{q}_{n}}
\end{equation}

再次, 由定理 \ref{thm:3} 可得

\begin{equation}
\label{eq:10}
\frac{{p}_{n - 2}}{{q}_{n - 1}} < \cdots < \frac{i{p}_{n - 1} + {p}_{n - 2}}{i{q}_{n - 1} + {q}_{n - 2}} < \frac{\left( {i + 1}\right) {p}_{n - 1} + {p}_{n - 2}}{\left( {i + 1}\right) {q}_{n - 1} + {q}_{n - 2}} < \cdots < \frac{{a}_{n}{p}_{n - 1} + {p}_{n - 2}}{{a}_{n}{q}_{n - 1} + {q}_{n - 2}} = \frac{{p}_{n}}{{q}_{n}} 
\end{equation}

所以必有唯一的 \({l}_{0}\left( {0 \leq {l}_{0} < {a}_{n}}\right)\) 使

\begin{equation}
\label{eq:11}
\frac{{l}_{0}{p}_{n - 1} + {p}_{n - 2}}{{l}_{0}{q}_{n - 1} + {q}_{n - 2}} \leq {2a} - \frac{{p}_{n - 1}}{{q}_{n - 1}} < \frac{\left( {{l}_{0} + 1}\right) {p}_{n - 1} + {p}_{n - 2}}{\left( {{l}_{0} + 1}\right) {q}_{n - 1} + {q}_{n - 2}} 
\end{equation}

即

\[
\alpha - \frac{{l}_{0}{p}_{n - 1} + {p}_{n - 2}}{{l}_{0}{q}_{n - 1} + {q}_{n - 2}} \geq \frac{{p}_{n - 1}}{{q}_{n - 1}} - \alpha > \alpha - \frac{\left( {{l}_{0} + 1}\right) {p}_{n - 1} + {p}_{n - 2}}{\left( {{l}_{0} + 1}\right) {q}_{n - 1} + {q}_{n - 2}}.
\]

将 \(\alpha = \frac{{\alpha }_{n}{p}_{n - 1} + {p}_{n - 2}}{{\alpha }_{n}{q}_{n - 1} + {q}_{n - 2}}\) 代入上式,并加整理,最后得

\[
\frac{1}{2}\left( {{\alpha }_{n} - \frac{{q}_{n - 2}}{{q}_{n - 1}}}\right) - 1 < {l}_{0} \leq \frac{1}{2}\left( {{\alpha }_{n} - \frac{{q}_{n - 2}}{{q}_{n - 1}}}\right) .
\]

由 \eqref{eq:9},\eqref{eq:10},\eqref{eq:11} 必有唯一的 \(l\left( {{l}_{0} + 1 \leq l \leq {a}_{n}}\right)\) 使

\[
\frac{\left( {l - 1}\right) {p}_{n - 1} + {p}_{n - 2}}{\left( {l - 1}\right) {q}_{n - 1} + {q}_{n - 2}} < \frac{P}{Q} \leq \frac{l{p}_{n - 1} + {p}_{n - 2}}{l{q}_{n - 1} + {q}_{n - 2}}.
\]

倘若等式不成立, 即若

\[
\frac{\left( {l - 1}\right) {p}_{n - 1} + {p}_{n - 2}}{\left( {l - 1}\right) {q}_{n - 1} + {q}_{n - 2}} < \frac{P}{Q} < \frac{l{p}_{n - 1} + {p}_{n - 2}}{l{q}_{n - 1} + {q}_{n - 2}}
\]

那就有
\begin{align*}
  &\frac{1}{Q\left\lbrack {\left( {l - 1}\right) {q}_{n - 1} + {q}_{n - 2}}\right\rbrack } \\
  \leq & \frac{P}{Q} - \frac{\left( {l - 1}\right) {p}_{n - 1} + {p}_{n - 2}}{\left( {l - 1}\right) {q}_{n - 1} + {q}_{n - 1}}\\
  <& \frac{l{p}_{n - 1} + {p}_{n - 2}}{l{q}_{n - 1} + {q}_{n - 2}} - \frac{\left( {l - 1}\right) {p}_{n - 1} + {p}_{n - 2}}{\left( {l - 1}\right) {q}_{n - 1} + {q}_{n - 2}}\\
  =& \frac{1}{\left\lbrack {\left( {l - 1}\right) {q}_{n - 1} + {q}_{n - 2}}\right\rbrack \left( {l{q}_{n - 1} + {q}_{n - 2}}\right) }.
\end{align*}

即得

\[
Q > l{q}_{n - 1} + {q}_{n - 2}
\]

但

\[
\alpha - \frac{P}{Q} > \alpha - \frac{l{p}_{n - 1} + {p}_{n - 2}}{l{q}_{n - 1} + {q}_{n - 2}} \geq 0,
\]

这和要求 \eqref{eq:6} 矛盾. 所以必有 \(\frac{P}{Q} = \frac{l{p}_{n - 1} + {p}_{n - 2}}{l{q}_{n - 1} + {q}_{n - 2}}\left( {{l}_{0} + 1 \leq l \leq {a}_{n}}\right)\) .

反之,这些分数的分母都适合 \({q}_{n - 1} < Q \leq {q}_{n}\) ,并且它们满足  \eqref{eq:6}. 因为假如 \(\frac{p}{q}\) 和 \(\alpha\) 的距离小于或等于 \(\frac{l{p}_{n - 1} + {p}_{n - 2}}{l{q}_{n - 1} + {q}_{n - 2}}\) 和 \(\alpha\) 的距离,那么 \(\frac{p}{q}\) 或者落在 \(\frac{{p}_{n}}{{q}_{n}}\) 和 \(\frac{{p}_{n - 1}}{{q}_{n - 1}}\) 之间,而由系\ref{cor:8.1}得 \(q > {q}_{n}\) ; 或者落在 \(\frac{l{p}_{n - 1} + {p}_{n - 2}}{l{q}_{n - 1} + {q}_{n - 2}}\) 和 \(\frac{{p}_{n}}{{q}_{n}}\) 之间,而有 \(k\left( {l < k \leq {a}_{n}}\right)\) 使

\[
\frac{k{p}_{n - 1} + {p}_{n - 2}}{k{q}_{n - 1} + {q}_{n - 2}} < \frac{p}{q} \leq \frac{\left( {k + 1}\right) {p}_{n - 1} + {p}_{n - 2}}{\left( {k + 1}\right) {q}_{n - 1} + {q}_{n - 2}}.
\]

由上面同样的证明,得到 \(q \geq \left( {k + 1}\right) {q}_{n - 1} + {q}_{n - 2} > l{q}_{n - 1} + {q}_{n - 2}\) . 所以总有

\[
q > i{q}_{n - 1} + {q}_{n - 2}
\]

也就是说 \(\frac{l{p}_{n - 1} + {p}_{n - 2}}{l{q}_{n - 1} + {q}_{n - 2}}\left( {{l}_{0} + 1 \leq l \leq {a}_{n}}\right)\) 满足 \eqref{eq:6}. 定理证完. 
\end{proof}

于是本节开始所提出的问题得到完全的解决.

\section{结束语}

我们在这里只挑选了少数容易说明的应用; 就问题的性质来说, 应用的范围是宽广的. 凡是几种周期的重遇或复迭. 都可能用到这一套数学; 而多种周期的现象, 经常出现于声波、光波、电波、水波和空气波的研究之中. 又如坝身每隔 \(a\) 分钟受某种冲击力,每隔 \(b\) 分钟受另一种冲击力,用这套数学可以确定大致每隔多少分钟最大的冲击力出现一次, 等等.

本书是为中学生写的. 和这有关的许多有趣的、更深入的问题。这里不谈了. 要想进一步了解的读者, 可以参考拙著 《数论导引》第十章.

为了迎接 1962 年的数学竞赛, 这本小书写得太匆忙了, 没有经过充分的修饰和考虑, 更没有预先和中学生们在一起共同研究一下, 希望读者、特别是中学教师和高中同学们多多提意见.

\rightline{1962 年春节完稿于从化温泉}

在完稿之后, 又改写了几次. 中国科技大学高等数学教研室副主任龚升同志曾就原稿提出了不少宝贵意见. 中国科学院自然科学史研究室严敦杰同志提供了有价值的历史资料. 而在改写过程中, 又曾得到吴方、徐诚浩、谢盛刚、李根道四位同志的帮助. 特别是吴方同志对第十二到十五书作了重大修改, 徐诚浩同志对有关天文的部分提了很多意见, 又北京天文馆刘麟仲同志提供了今年春节 “五星联珠, 七曜同宫” 现象的图象, 对于以上诸同志的帮助, 一并在此致谢.

\rightline{于中国科学技术大学}

\rightline{1962 年四月八日}

\section{附录:祖冲之简介}

祖冲之, 字文远, 生于公元 429 年, 卒于公元 500 年. 他的粗籍是范阳郡蓟县, 就是现在的河北省涞源县。 他是南北朝时代南朝宋齐之间的一位杰出的科学家. 他不仅是一位数学家, 同时还通晓天文历法、机械制造、音乐, 并且是一位文学家.

在机械制造方面, 他重造了指南车, 改进了水碓磨, 创制了一艘 “千里船”. 在音乐方面, 人称他 “精通“钟律”, 独步一时’。在文学方面,他着有小说《述异记》十卷.

祖家世世代代都对天文历法有研究, 他比较容易接触到数学的文献和历法资料, 因此他从小对数学和天文学就发生兴趣. 用他自己的话来说, 他从小就“专攻数术, 搜炼古今”. 这“搜”、“炼”两个字, 刻划出他的治学方法和精神.

“搜”表明他不但阅读了祖辈相传的文献和资料, 还主动去寻找从远古到他所生活的时代的各项文献和观测纪录,也就是说他尽量吸收了前人的成就. 而更重要的还在“炼”字上, 他不仅阅读了这些文献和数据, 并且做过一些“由表及里, 去芜存精”的工作, 把自己所搜到的资料经过消化, 据为己有. 最具体的例子是注解了我国历史上的名著《九章算术》.

他广博地学习和消化了古人的成就和古代的资料, 但是他不为古人所局囿, 他决不“虚推古人”, 这是另一个可贵的特点. 例如他接受了刘徽算圆周率的方法, 但是他并不满足于刘徽的 结果  \(3.14\frac{64}{125}\) ,他进一步计算,算到圆内接正 1536 边形, 得出圆周率 3.1416. 但是他还不满足于这一结果, 又推算下去, 得出

\[
{3.1415926} < \pi < {3.1415927}\text{.}
\]

这一结果的重要意义在于指出误差范围. 大家不要低估这个工作, 它的工作量是相当巨大的. 至少要对 9 位数字反复进行 130 次以上的各种运算, 包括开方在内. 即使今天我们用纸笔来算,也绝不是一件轻松的事,何况古代计算还是用算筹 (小竹棍) 来进行的呢? 这需要怎样的细心和毅力啊! 他这种严谨不苟的治学态度, 不怕复杂计算的毅力, 都是值得我们学习的.

他在历法方面测出了地球绕日一周的时间是 365.24231481 日, 跟现在知道的数据 365.2422 对照, 知道他的数值准到小数第三位. 这当然是由于受当时仪器的限制; 根据这个数字, 他提出了把农历的19年 7 闰改为 391 年 144 闰的主张。 这一论断虽有它由于测量不准的局限性, 但是他的数学方法是正确的 (读者可以根据本书的论述来判断这一建议的精密度: \(\frac{{10.8750}}{{29.5306}} \approx {0.36826}\) 只能准到四位, \(\frac{7}{19} = {0.3684}\) . \(\left. {\frac{144}{391} = {0.36829},\frac{116}{315} = {0.36825}}\right)\) ,

他这种勤奋实践、不怕复杂计算和精细测量的精神, 正如他所说的“亲量圭尺, 躬察仪漏, 目尽毫厘, 心穹筹算”. 由于有这样的精神, 他发现了当时历法上的错误, 因此着手编制出新的历法, 这是当时最好的历法. 在公元 462 年 (刘宋大明六年), 他上表给皇帝刘骏, 请讨论颁行, 定名为“大明历”.

新的历法遭到了戴法兴的反对. 戴是当时皇帝的宠幸人物, 百官惧怕戴的权势, 多所附和. 戴法兴认为 “古人制章” “万世不易”, 是“不可革”的, 认为天文历法“非凡夫所测”. 甚至于骂祖冲之是“诬天背经”, 说“非冲之浅虑, 妄可穿凿”的. 祖冲之并没有为这权贵所吓倒,他写了一篇《驳议》,说“愿闻显据, 以窍理实”, 并表示了 “浮词虚贬, 窃非所惧”的正确立场.

这场斗争祖冲之并没有得到胜利, 一直到他死后, 由于他的儿子祖暅的再三坚持, 经过了实际天象的检验, 在公元 510 年(梁天监九年)才正式颁行. 这已经是祖冲之死后的第十个年头了.

祖仲之的数学专着《缀朮》已经失传。《隋书》中写道: “……祖冲之……所著书, 名为缀术, 学官莫能究其深奥, 是故废而不理." 这是我们数学史上的一个重大损失.

祖冲之虽已去世一千四百多年, 但他的广泛吸收古人成就而不为其所拘泥、艰苦劳动、勇于创造和敢于坚持真理的精神, 仍旧是我们应当学习的榜样.

\end{document}

%%% Local Variables:
%%% mode: latex
%%% TeX-master: t
%%% End:
