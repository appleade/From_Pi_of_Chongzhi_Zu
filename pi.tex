% -*- coding: utf-8 -*-
\documentclass{ctexart}
\usepackage{amsmath, amssymb, amsthm, bm, ulem, hyperref, graphicx}
\usepackage{paracol}
\usepackage[margin = 1 in]{geometry}

\usepackage{tipa}
\makeatletter
\newcommand{\cemph}[1]{\@tfor\i:=#1\do{\textsubdot{\i}}}
\makeatother

\title{从祖冲之的圆周率谈起(节选)}
\author{华罗庚}
%\date{1964年2月}
\date{}

\newtheorem{theorem}{\indent 定理}[section]
\newtheorem{lemma}[theorem]{\indent 引理}
\newtheorem{proposition}[theorem]{\indent 命题}
\newtheorem{corollary}[theorem]{\indent 推论}
\newtheorem{definition}{\indent 定义}[section]
\newtheorem{example}{\indent 例}[section]
\newtheorem{remark}{\indent 注}[section]
\newenvironment{solution}{\begin{proof}[\indent\bf 解]}{\end{proof}}
\renewcommand{\proofname}{\indent\bf 证明}

\renewcommand\theenumi{\roman{enumi}}

\NewDocumentCommand{\dotunderline}{m}{%
  \settowidth{\dimen0}{#1}%
  \makebox[\dimen0]{%
    \ooalign{%
      #1\cr
      \hidewidth\ensuremath{\underbrace{\rule[-1ex]{\dimen0}{0.4pt}}_{\makebox[0pt]{\hspace{0.1em}$\cdot$\hspace{0.1em}}}\hidewidth}%
    }%
  }%
}

%\newcommand{\cplus}{\raisebox{-1.1ex}{+}}
\newcommand{\cplus}{\oplus}


\begin{document}

\maketitle
% \section{祖冲之的约率 $\frac{22}{7}$ 和 密率 $\frac{355}{113}$}
% 祖冲之是我国伟大的数学家。他生于公元429年,卒于公元500年。他的儿子祖暅和他的孙子祖皓,也都是数学家,善算历。

% 关于圆周率$\pi$,祖冲之的贡献有二:
% \begin{enumerate}
% \item\label{item:1} $3.1415926<\pi<3.1415927$;
% \item\label{item:2} 他用 $\frac{22}{7}$ 作为约率,$\frac{355}{113}$ 作为密率。
% \end{enumerate}
% 这些结果是刘微割圆术之后的重要发展。刘徽从圆内接
% 正六边形起算,令边数一倍一倍地增加,即12、24、48,96、 ……
% 1536, …… ,因而逐个算出六边形、十二边形、二十四边形 ……
% 的面积,这些数植逐步地逼近圆周率。刘微方法的特点,是得
% 出一批一个大于一个的数值,这样来一步一步地逼近圆周率。
% 这方法是可以无限精密地逼近圆周率的,但每一次都比圆周率小。

% 祖冲之的结果(~\ref{item:1})从上下两方面指出了圆周率的误差范
% 围。这是大家都容易看到的事实,因此在这本小书中不预备
% 多讲。我只准备着重地谈一谈结果(~\ref{item:2})。在谈到$\frac{355}{133}$的时候,一定能从
% \begin{equation*}
% \frac{355}{113}  = 3.1415929\cdots
% \end{equation*}
% 看出,他所提出的 $\frac{355}{113}$ 惊人精密的接近于圆周率,准确到六位小数。也有人会指出这一发现比欧洲人早了一千年。因为德
% 国人奥托(Valenlinus Otto)在1573年才发现这个分数。如果更深入地想一下,就会发现$\frac{22}{7}$和$\frac{355}{113}$的意义还远不止这
% 些。有些人认为那时的人们喜欢用分数来计算。这样看问题未免太简单了。其实其中孕育着不少道理,这道理可以用来推算天文上的很多现象。无怪乎祖冲之祖孙三代都是算历的专家。这个约率和密率,提出了“用有理数最佳逼近实数”的问題。“逼近”这个概念在近代数学中是十分重要的。

\section{人造行星将于2113年又接近地球}

%我们暂且把“用有理数最佳逼近实数”的问题放一放,而再提一个事实:

% \noindent
% \begin{minipage}[t]{0.3\textwidth} % Left column: 30% of the text width
%     \includegraphics[width = 0.8 \textwidth]{sun.png}
% \end{minipage}
% \begin{minipage}[t]{0.6\textwidth} % Right column: 65% of the text width
%     
% \end{minipage}

\columnratio{0.3}
\begin{paracol}{2} % 3:1 ratio (left:right)
  \begin{leftcolumn}
  \includegraphics[width= 0.8 \linewidth]{sun.png}
  \end{leftcolumn}

\begin{rightcolumn}
  1959年苏联第一次发射了一个(绕太阳公转的)人造行星,报上说:苏联某专家算出,五年后这个人造行星又将接近地球,在2113年又将非常接近地球,这是怎样算出来的?难不难,深奥不深奥?我们中学生能懂不能懂?我说能懂的:不需要专家,中学生是可以学懂这个方法的。
\end{rightcolumn}

\end{paracol}

先看为什么五年后这个人造行星会接近地球。报上登过这个人造行星绕太阳一周的时间是450天。如果以地球绕日一周360天算,地球走五圈和人造行星走四圈不都是1800天吗?因此五年后地球和人造行星将相互接近。至于为什么在2113年这个人造行星和地球又将非常接近? 我们将在第三节中说明。

再看五圈是怎样算出来的。 任何中学生都会回答:这是由于约分:
\begin{equation*}
\frac{450}{360} = \frac{5}{4}
\end{equation*}
而得来的,或者这是求$450$和$360$的最小公倍数而的来的。它们的最小公倍数是$1800$,而$\frac{1800}{360} = 5, \frac{1800}{450} = 4$;也就是当地球绕太阳$5$圈时,人造行星恰好回到了原来的位置。求最小公倍数在这儿找到了用场。在进入下节介绍辗转相除法之前,我们再说一句,地球绕太阳并不是$360$天一周,而是$365 \frac{1}{4}$天。因而仅仅学会求最小公倍数法还不足以应付这一问题,还需更上一层楼。

【停!】
\section{辗转相除法和连分数}
我们还是循序渐进吧。先从简单的(原来在小学或初中一年级学到的)辗转相除法讲起。但我们采用较高的形式,采用学过代数学的同学能理解的形式。

给两个正整数$a$和$b$,用$b$除$a$得商$a_0$,余数$r$,写成式子
\begin{equation}
\label{eq:3}
a = a_0b + r \quad 0\le r < b
\end{equation}
\cemph{这是最基本的式子}。如果$r$等于$0$,那么$b$可以除尽$a$(记作$b|a$),而$a,b$的最大公约数就是$b$。

如果$r\ne 0$,再用$r$除$b$,得商$a_1$,余数$r_1$,即
\begin{equation}
\label{eq:1}
b = a_1 r + r_1 \quad 0\le r_1<r
\end{equation}
如果$r_1 = 0$,那么$r$除尽$b$,由~\eqref{eq:3}它也除尽$a$。又任何一个除尽$a$和$b$的数,由~\eqref{eq:1}也一定除尽$r$。因此,$r$是$a,b$的最大公约数。

【停!】

如果$r_1\ne 0$,用$r_1$除$r$,得商$a_2$,余数$r_2$,即
\begin{equation}
\label{eq:2}
r = a_2r_1 + r_2
\end{equation}
如果$r_2=0$,那么由~\eqref{eq:1} $r_1$是$b,r$的公约数,由~\eqref{eq:3}它也是$a,b$的公约数。反之,如果一数除得尽$a,b$,那由~\eqref{eq:3}它一定除得尽$b,r$,由~\eqref{eq:1}它一定除得尽$r,r_1$,所以$r_1$是$a,b$的最大公约数。

如果$r_2\ne 0$,再用$r_2$除$r_1$,如法进行。由于$b>r>r_1>r_2>\cdots(\ge0)$逐步小下来,因此经过有限步骤后一定可以找出$a,b$的最大公约数来(最大公约数可以是$1$)。这就是 \textbf{辗转相除法},或称 \textbf{欧几里得算法}。这个方法是我们这本小册子的灵魂。

【停!】

\begin{example}
  求$360$和$450$的最大公约数。
  \begin{align*}
    450 = & 1 \times 360 + 90\\
    360 = & 4 \times 90
  \end{align*}
  所以$90$是$360,450$的最大公约数。由于最小公倍数等于两数相乘再除以最大公约数,因此二数的最小公倍数等于
\begin{equation*}
360\times450\div 90 = 1800
\end{equation*}
因而得出上节的结果。
\end{example}

\begin{example}
  \label{sec:2}
  求 $42897$和 $18644$的最大公约数。
  \begin{align*}
    42897 = & 2 \times 18644 + 5609\\
    18644 = & 3 \times 5609 + 1817\\
    5609 = & 3 \times 1817 + 158\\
    1817 = & 11 \times 158 + 79\\
    158 = & 2 \times 79
  \end{align*}
因此最大公约数等于 $79$。
\end{example}

计算的草式如下:

\begin{table}[h]
 \centering
\begin{tabular}{r|l|l}
42897 &    &       \\
$-$ 37288 & 2  & 18644 \\ \cline{1-1}
5609  & 3  & 16827 \\ \cline{3-3} 
5451  & 3  & 1817  \\ \cline{1-1}
158   & 11 & 1738  \\ \cline{3-3} 
158   & 2  & 79    \\ \cline{1-1}
0     &    &      
\end{tabular}
\end{table}

【停!】

例 \ref{sec:2} 的计算也可以写成为
\begin{align*}
  \frac{42897}{18644} = & 2 + \cfrac{5609}{18644} = 2 + \cfrac{1}{\cfrac{18644}{5609}}\\
  = & 2 + \cfrac{1}{3 + \cfrac{1817}{5609}} = 2 + \cfrac{1}{3+\cfrac{1}{3 + \cfrac{158}{1817}}}\\
  = & 2 + \cfrac{1}{3 + \cfrac{1}{3 + \cfrac{1}{11+\cfrac{79}{158}}}}\\
  = & 2 + \cfrac{1}{3 + \cfrac{1}{3 + \cfrac{1}{11+\cfrac{1}{2}}}}
\end{align*}
这样的繁分数称为 \textbf{连分数}。

【停!】

为了节省篇幅,我们把它写成
\begin{equation*}
2 + \frac{1}{3} \cplus \frac{1}{3} \cplus \frac{1}{11} \cplus \frac{1}{2}
\end{equation*}
注意$2,3,3,11,2$都是草式中间一行的数字。倒算回去,得
\begin{align*}
  2 + \frac{1}{3} \cplus \frac{1}{3} \cplus \frac{1}{11} \cplus \frac{1}{2} = & 2 + \frac{1}{3} \cplus \frac{1}{3} \cplus \frac{2}{23}\\
  = & 2 + \frac{1}{3} \cplus \frac{23}{71} = 2 + \frac{71}{236} = \frac{543}{236}
\end{align*}
(其中$\oplus$ 表示连分数加法。例如,$\frac{1}{2}\oplus \frac{1}{3} = \frac{1}{2 + (1/3)} = \frac{1}{7/3} = \frac{3}{7}$。)这就是原来分数的既约分数。

【停!】

依次截断得
$$
2,\quad 2+ \frac{1}{3} = \frac{7}{3}, \quad 2 + \frac{1}{3} \cplus \frac{1}{3} = \frac{23}{10}, \quad 2 + \frac{1}{3} \cplus \frac{1}{3} \cplus \frac{1}{11} = \frac{260}{113} 
$$
这些分数称为$\frac{543}{236}$的\textbf{渐近分数}。我们看到第一个渐近分数比 $\frac{543}{236}$小,第二个渐近分数比它大,第三个又比它小,……

为什么叫做渐近分数?例如,我们看一下分母不超过$10$的分数和$\frac{543}{236}$相接近的情况。

分母是$1,2,3,4,5,6,7,8,9,10$,而最接近于$\frac{543}{236}$的分数是
\begin{equation*}
\frac{2}{1}, \frac{5}{2}, \frac{7}{3}, \frac{9}{4}, \frac{12}{5}, \frac{14}{6}, \frac{16}{7}, \frac{19}{8}, \frac{21}{9}, \frac{23}{10}
\end{equation*}
取二位小数,它们分别等于
\begin{equation*}
2.00, 2.50, 2.33, 2.25, 2.40, 2.33, 2.29, 2.38, 2.33, 2.30.
\end{equation*}
和$\frac{543}{236} = 2.30$ 相比较,可以发现其中有几个特殊的既约分数
\begin{equation*}
\frac{2}{1}, \frac{5}{2}, \frac{7}{3}, \frac{16}{7}, \frac{23}{10}
\end{equation*}
在上面这一列数中,这几个数比它们以前的数都更接近于$\frac{543}{236}$。而其中$\frac{2}{1}, \frac{7}{3}, \frac{23}{10}$ 都是由连分数截断算出来的数,即它们都是渐近分数。

【停!】

我们现在再证明:分母小于$113$的分数里面,没有一个比$\frac{260}{113}$更接近于$\frac{543}{236}$了,要证明这点很容易,首先
\begin{equation*}
\left|\frac{543}{236} - \frac{260}{113}\right| = \frac{1}{236\times113}
\end{equation*}
命$\frac{a}{b}$是任意分母小于$113$的分数,那么
\begin{equation*}
\left|\frac{543}{236} - \frac{a}{b}\right| = \frac{|543b - 236a|}{236\times b} \ge \frac{1}{236\times b} > \frac{1}{236\times113}
\end{equation*}

【停!】

\section{答第一节的问}
现在我们来回答第一节里的问题:怎样算出人造行星2113年又将非常接近地球?

人造卫星绕日一周需 $450$ 天,地球绕日一周是 $365 \frac{1}{4}$天。如果以$\frac{1}{4}$天做单位,那么人造行星和地球绕日一周的时间各为 $1800$ 和 $1461$ 个单位。如上节所讲的方法,
\begin{table}[h]
  \centering
\begin{tabular}{r|l|l}
1800 &   &      \\
1461 & 1 & 1461 \\ \cline{1-1}
339  & 4 & 1356 \\ \cline{3-3} 
315  & 3 & 105  \\ \cline{1-1}
24   & 4 & 96   \\ \cline{3-3} 
18   & 2 & 9    \\ \cline{1-1}
6    & 1 & 6    \\ \cline{3-3} 
6    & 2 & 3    \\ \cline{1-1}
0    &   &     
\end{tabular}
\end{table}

即得连分数
\begin{equation*}
1 + \frac{1}{4} \cplus \frac{1}{3} \cplus \frac{1}{4} \cplus \frac{1}{2} \cplus \frac{1}{1} \cplus \frac{1}{2}
\end{equation*}

由此得渐近分数
\begin{align*}
  1, & 1 + \frac{1}{4} = \frac{5}{4}, \quad 1 + \frac{1}{4} \cplus \frac{1}{3} = \frac{16}{13}, \quad 1 + \frac{1}{4} \cplus  \frac{1}{3} \cplus \frac{1}{4} = \frac{69}{56},\\
  & 1 + \frac{1}{4} \cplus  \frac{1}{3} \cplus  \frac{1}{4} \cplus  \frac{1}{2} = \frac{154}{125}, \cdots
\end{align*}

【停!】

第一个渐近分数说明了地球5圈,人造卫星4圈,即5年后人造卫星和地球接近。但地球16圈,人造卫星13圈更接近些;地球69圈,人造卫星56圈还要接近些;而地球154圈,人造卫星125圈又要更接近些。这就是报上所登的苏联专家所算出的数字了,这也就是在
\begin{equation*}
1959 + 154 = 2113
\end{equation*}
年,人造卫星将非常接近地球的道理。

当然,由于连分数还可以继续做下去,所以我们可以更精密地算下去;但是因为$450$天和$365 \frac{1}{4}$ 天这两个数字本身并不很精确,所以再算下去也没有太大的必要了。但读者不妨作为习题再算上一项。

【停!】

\section{习题}
\begin{enumerate}
\item 请根据你的理解,简述为什么苏联卫星将于2113年接近地球。
\item 在刚刚的阅读过程中,你认为自己使用了哪些阅读策略?
\end{enumerate}









\end{document}
%%% Local Variables:
%%% mode: latex
%%% TeX-master: t
%%% End:
